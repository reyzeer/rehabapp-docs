%%%%%%%%%%%%%%%%%%%%%%%%%%%%%%%%%%%%%%%%%%%%%%%%%%%
%% Author: Zbigniew Czarnecki 209909             %%
%% Latex scheme parent fork: tasklife-docs       %%
%% Latex scheme version: 0.2                     %%
%% + podzial na kolumny							 %%
%% + szablon wymagania funkcjonalne              %%
%% + szablon wymagania poza funkcjonalne         %%
%%%%%%%%%%%%%%%%%%%%%%%%%%%%%%%%%%%%%%%%%%%%%%%%%%%

\documentclass[a4paper, 11pt]{article}
\usepackage[margin=2.5cm]{geometry}

\usepackage[T1]{fontenc}
\usepackage[utf8]{inputenc}
\usepackage[english, polish]{babel}

%wciecie pierwszego wiersza
\usepackage{indentfirst}
\setlength{\parindent}{1cm}

%ładne linki (nie rzuca się o podkreślenia i procenty)
\usepackage{hyperref}

%manipulacje kolumną w tabeli (pogrubienie tekstu w całej kolumnie)
\usepackage{array}

%tabele na do maksymalnej szerokosci strony
\usepackage{tabularx}

%dodawanie plików graficznych
\usepackage{graphicx}

%lepsza czcionka do latexa
\usepackage{times}

%podział na kolumny
\usepackage{multicol}

%%% Schematy

%tłusty tekst, do lewej
\newcolumntype{R}[1]{>{\raggedright\bfseries}p{#1}}
%do lewej
\newcolumntype{L}[1]{>{\raggedright}p{#1}}

%% Wymagania funckjonalne

% 
\def\funcreq#1#2#3#4#5#6#7#8#9{%
	\def\funcreqcontinued##1{%
		\vspace{0.2cm}
		\begin{tabularx}{\textwidth}{|R{4cm}|L{11.1cm}|}
			\hline
			Nr wymagania & #1 \tabularnewline \hline
			Nazwa & #2 \tabularnewline \hline
			Opis & #3 \tabularnewline \hline
			Przesłanka & #4 \tabularnewline \hline
			Ograniczenia i warunki & #5 \tabularnewline \hline
			Dane wejściowe & #6 \tabularnewline \hline
			Źródło danych & #7 \tabularnewline \hline
			Wynik & #8 \tabularnewline \hline
			Priorytet & #9 \tabularnewline \hline
			Kryterium do spełnienia & ##1 \tabularnewline \hline
		\end{tabularx}
		\vspace{0.2cm}
	}%
	\funcreqcontinued%
}

%Przyklad: \funcreq{Nr wymagania}{Nazwa}{Opis}{Przesłanka}{Ograniczenia i warunki}{Dane wejściowe}{Źródło danych}{Wynik}{Priorytet}{Kryterium do spełnienia}

%\funcreq{
%		1
%	}{
%		Nazwa
%	}{
%		Opis
%	}{
%		Przesłanka
%	}{
%		Ograniczenia i warunki
%	}{
%		Dane wejściowe
%	}{
%		Źródło danych
%	}{
%		Wynik
%	}{
%		Priorytet
%	}{
%		Kryterium do spełnienia
%}

%% Wymagania poza funkcjonalne

\newcommand{\nonefuncreq}[5]{
	\vspace{0.2cm}
	\begin{tabularx}{\textwidth}{|R{4cm}|L{11.1cm}|}
		\hline
		Nr wymagania & #1 \tabularnewline \hline
		Nazwa & #2 \tabularnewline \hline
		Opis & #3 \tabularnewline \hline
		Przesłanka & #4 \tabularnewline \hline
		Kryterium do spełnienia & #5 \tabularnewline \hline
	\end{tabularx}
	\vspace{0.2cm}
}

%Przyklad: \nonefuncreq{Nr wymagania}{Nazwa}{Opis}{Przesłanka}{Kryterium do spełnienia}

%\nonefuncreq{
%		1
%	}{
%		Nazwa
%	}{
%		Opis
%	}{
%		Przesłanka
%	}{
%		Kryterium do spełnienia
%}

\begin{document}

\begin{titlepage}
	
	\begin{center}
		\huge{\uppercase{\textbf{Politechnika Wrocławska}}}
	\end{center}

	\begin{flushright}
		Wrocław, 30 październik 2017
	\end{flushright}

	\vspace{2cm}
	\begin{flushleft}
		\textbf{\\
			Wydział Elektroniki\\
			Kierunek: Informatyka\\
			Semestr: 2017/18 ZIMA\\
			Prowadzący: dr inż. Konrad Jackowski \\
		}
	\end{flushleft}
	
	\vspace{3cm}
	\begin{center}
		\textbf{
			\huge{\uppercase{Projektowanie telemedycznych systemów internetowych i~mobilnych}}\\
			\LARGE{projekt}
		}\\
		\huge{Temat: Aplikacja do zarządzania dniem pracy specjalistów oraz rejestracji pacjentów w~zakładzie rehabilitacji}
	\end{center}
	
	\vspace{5cm}
	\begin{multicols}{2}
		\begin{tabularx}{\textwidth}{ll}
			Zbigniew Czarnecki & 209909 \\
			Maksymilian Iwanow & 209946 \\
			Mateusz Ligus & 209939 \\
		\end{tabularx}
		\columnbreak
		\begin{tabularx}{\textwidth}{ll}
			Marcin Pawłowski & 218441 \\
			Jakub Sanecki & 210016 \\		
			Piotr Wolf & 210036 \\
		\end{tabularx}
	\end{multicols}

\end{titlepage}

\tableofcontents
\section{Specyfikacja wymagań}

\subsection{Czynniki sterujące projektem}

\subsubsection{Cele projektu}

\begin{itemize}
	\item Wdrożenie aplikacji do zarządzania dniem pracy specjalistów
	i~przypisywania im sprzętu rehabilitacyjnego na określoną porę,
	\begin{itemize}
		\item 30\% zwiększenie efektywności pracy poprzez usprawnienie systemu przypisywania specjalistom dostępu do sprzętu rehabilitacyjnego, maksymalnie pokrywając ich czas pracy zajęciami,
	\end{itemize}
	\item Wdrożenie systemu do rejestracji pacjentów,
	\begin{itemize}
		\item 50\% zmniejszenie kosztów obsługi rejestracji pacjentów, poprzez automatyzację tej procedury, co umożliwi redukcję etatów pracowników recepcji.
	\end{itemize}
\end{itemize}

\subsubsection{Użytkownicy produktu}

\begin{itemize}
	\item Klient
	\begin{itemize}
		\item Obsługa zakładu rehabilitacyjnego, administrująca czasem pracy specjalistów jak i~dostępem do sal.
	\end{itemize}
	\item Użytkownicy
	\begin{itemize}
			\item Wszyscy potencjalni klienci rejestrujący się na zajęcia rehabilitacyjne.
	\end{itemize}
\end{itemize}


\subsection{Czynniki sterujące projektem}

\subsubsection{Cele projektu}

\begin{itemize}
	\item Wdrożenie aplikacji do zarządzania dniem pracy specjalistów
	i~przypisywania im sprzętu rehabilitacyjnego na określoną porę,
	\begin{itemize}
		\item 30\% zwiększenie efektywności pracy poprzez usprawnienie systemu przypisywania specjalistom dostępu do sprzętu rehabilitacyjnego, maksymalnie pokrywając ich czas pracy zajęciami,
	\end{itemize}
	\item Wdrożenie systemu do rejestracji pacjentów,
	\begin{itemize}
		\item 50\% zmniejszenie kosztów obsługi rejestracji pacjentów, poprzez automatyzację tej procedury, co umożliwi redukcję etatów pracowników recepcji.
	\end{itemize}
\end{itemize}

\subsubsection{Użytkownicy produktu}

\begin{itemize}
	\item Klient
	\begin{itemize}
		\item Obsługa zakładu rehabilitacyjnego, administrująca czasem pracy specjalistów jak i~dostępem do sal.
	\end{itemize}
	\item Użytkownicy
	\begin{itemize}
			\item Wszyscy potencjalni klienci rejestrujący się na zajęcia rehabilitacyjne.
	\end{itemize}
\end{itemize}

\subsection{Ograniczenia projektu}

\subsubsection{Ograniczenia wynikające z natury i~okoliczności projektu}

\begin{itemize}
	\item Wszystkie dane klientów muszą podlegać ochronie zgodnie z ustawą o~ochronie danych osobowych (Dz.U. 1997 Nr 133 poz. 883),
	\item Aplikacja ma zostać wykonana w formie aplikacji webowej,
	\item Do działania aplikacji konieczne jest połączenie z internetem,
	\item Wszystkie prace projektowe powinny zostać zaplanowane tak, aby umożliwić uruchomienie i~pełne funkcjonowanie systemu do 1~grudnia~2017~roku,
Brak funduszy, ponieważ jest to projekt studencki.
	%TODO
	\item Dodałbym jeszcze ograniczenie związane z technologią czyli Yii2 jak mniemam. Zbychu bardziej sie znasz to możesz ładnie sformułować.
\end{itemize}

\subsubsection{Konwencje nazewnicze i~definicje}
%TODO !!!!! Tutaj trzeba wstawić wszystkie akronimy i skróty myślowe które popełnimy w tym dokumencie !!!
\begin{itemiz}
	\item \textit{Aplikacja webowa} - program znajdujący się na serwerze komunikujący się z użytkownikiem za pomocą przeglądarki internetowej,
	\item \textit{Przeglądarka internetowa} - program służący do wyświetlania aplikacji webowych,
	\item \textit{Logowanie} - proces autoryzacji użytkownika polegający podaniu identyfikatora i~hasła,
\end{itemize}

\subsubsection{Fakty i~założenia powiązane z~projektem}
\begin{itemize}
	\item Witryna ma być zaprojektowana w taki sposób, aby nie sprawiać trudności podczas rejestracji pacjentów.
	\item Należy założyć, że z~witryny mogą korzystać zarówno osoby starsze jak i~młodsze, interfejs powinien być intuicyjny aby obsłużyć go mogły osoby niedoświadczone jak i~aby osoby obyte z~tego typu narzędziami mogły szybko wykonać chciane operacje. %TODO paskudny język
\end{itemize}

%\include{chapters/wstep}
%\include{chapters/projekt}
%\include{chapters/opis_implementacji}
%\include{chapters/bibliografia}
%\include{chapters/inwentaryzacha_sprzetu_i_infrastuktury_dostepnej_w_przedsiborstwie}
%\include{chapters/analiza_potrzeb_uzytkownika}

\end{document}
