%%%%%%%%%%%%%%%%%%%%%%%%%%%%%%%%%%%%%%%%%%%%%%%%%%%
%% Author: Zbigniew Czarnecki 209909             %%
%% Latex scheme parent fork: tasklife-docs       %%
%% Latex scheme version: 0.2                     %%
%% + podzial na kolumny							 %%
%% + szablon wymagania funkcjonalne              %%
%% + szablon wymagania poza funkcjonalne         %%
%%%%%%%%%%%%%%%%%%%%%%%%%%%%%%%%%%%%%%%%%%%%%%%%%%%

\documentclass[a4paper, 11pt]{article}
\usepackage[margin=2.5cm]{geometry}

\usepackage[T1]{fontenc}
\usepackage[utf8]{inputenc}
\usepackage[english, polish]{babel}

%wciecie pierwszego wiersza
\usepackage{indentfirst}
\setlength{\parindent}{1cm}

%ładne linki (nie rzuca się o podkreślenia i procenty)
\usepackage{hyperref}

%manipulacje kolumną w tabeli (pogrubienie tekstu w całej kolumnie)
\usepackage{array}

%tabele na do maksymalnej szerokosci strony
\usepackage{tabularx}

%dodawanie plików graficznych
\usepackage{graphicx}

%lepsza czcionka do latexa
\usepackage{times}

%podział na kolumny
\usepackage{multicol}

%%% Schematy

%tłusty tekst, do lewej
\newcolumntype{R}[1]{>{\raggedright\bfseries}p{#1}}
%do lewej
\newcolumntype{L}[1]{>{\raggedright}p{#1}}

%% Wymagania funckjonalne

% 
\def\funcreq#1#2#3#4#5#6#7#8#9{%
	\def\funcreqcontinued##1{%
		\vspace{0.2cm}
		\begin{tabularx}{\textwidth}{|R{4cm}|L{11.1cm}|}
			\hline
			Nr wymagania & #1 \tabularnewline \hline
			Nazwa & #2 \tabularnewline \hline
			Opis & #3 \tabularnewline \hline
			Przesłanka & #4 \tabularnewline \hline
			Ograniczenia i warunki & #5 \tabularnewline \hline
			Dane wejściowe & #6 \tabularnewline \hline
			Źródło danych & #7 \tabularnewline \hline
			Wynik & #8 \tabularnewline \hline
			Priorytet & #9 \tabularnewline \hline
			Kryterium do spełnienia & ##1 \tabularnewline \hline
		\end{tabularx}
		\vspace{0.2cm}
	}%
	\funcreqcontinued%
}

%Przyklad: \funcreq{Nr wymagania}{Nazwa}{Opis}{Przesłanka}{Ograniczenia i warunki}{Dane wejściowe}{Źródło danych}{Wynik}{Priorytet}{Kryterium do spełnienia}

%\funcreq{
%		1
%	}{
%		Nazwa
%	}{
%		Opis
%	}{
%		Przesłanka
%	}{
%		Ograniczenia i warunki
%	}{
%		Dane wejściowe
%	}{
%		Źródło danych
%	}{
%		Wynik
%	}{
%		Priorytet
%	}{
%		Kryterium do spełnienia
%}

%% Wymagania poza funkcjonalne

\def\nonefuncreq#1#2#3#4#5#6#7#8#9{%
	\def\nonefuncreqcontinued##1{%
		\vspace{0.2cm}
		\begin{tabularx}{\textwidth}{|R{4cm}|L{11.1cm}|}
			\hline
			Nr wymagania & #1 \tabularnewline \hline
			Nazwa & #2 \tabularnewline \hline
			Opis & #3 \tabularnewline \hline
			Przesłanka & #4 \tabularnewline \hline
			Kryterium do spełnienia & ##1 \tabularnewline \hline
		\end{tabularx}
		\vspace{0.2cm}
	}%
	\nonefuncreqcontinued%
}

\newcommand{\}{def}

%Przyklad: \nonefuncreq{Nr wymagania}{Nazwa}{Opis}{Przesłanka}{Kryterium do spełnienia}

%\nonefuncreq{
%		1
%	}{
%		Nazwa
%	}{
%		Opis
%	}{
%		Przesłanka
%	}{
%		Kryterium do spełnienia
%}

\begin{document}

\begin{titlepage}
	
	\begin{center}
		\huge{\uppercase{\textbf{Politechnika Wrocławska}}}
	\end{center}

	\begin{flushright}
		Wrocław, 30 październik 2017
	\end{flushright}

	\vspace{2cm}
	\begin{flushleft}
		\textbf{\\
			Wydział Elektroniki\\
			Kierunek: Informatyka\\
			Semestr: 2017/18 ZIMA\\
			Prowadzący: dr inż. Konrad Jackowski \\
		}
	\end{flushleft}
	
	\vspace{3cm}
	\begin{center}
		\textbf{
			\huge{\uppercase{Projektowanie telemedycznych systemów internetowych i~mobilnych}}\\
			\LARGE{projekt}
		}\\
		\huge{Temat: Aplikacja do zarządzania dniem pracy specjalistów oraz rejestracji pacjentów w~zakładzie rehabilitacji}
	\end{center}
	
	\vspace{5cm}
	\begin{multicols}{2}
		\begin{tabularx}{\textwidth}{ll}
			Zbigniew Czarnecki & 209909 \\
			Maksymilian Iwanow & 209946 \\
			Mateusz Ligus & 209939 \\
		\end{tabularx}
		\columnbreak
		\begin{tabularx}{\textwidth}{ll}
			Marcin Pawłowski & 218441 \\
			Jakub Sanecki & 210016 \\		
			Piotr Wolf & 210036 \\
		\end{tabularx}
	\end{multicols}

\end{titlepage}

\tableofcontents
\section{Specyfikacja wymagań}

\subsection{Czynniki sterujące projektem}

\subsubsection{Cele projektu}

\begin{itemize}
	\item Wdrożenie aplikacji do zarządzania dniem pracy specjalistów
	i~przypisywania im sprzętu rehabilitacyjnego na określoną porę,
	\begin{itemize}
		\item 30\% zwiększenie efektywności pracy poprzez usprawnienie systemu przypisywania specjalistom dostępu do sprzętu rehabilitacyjnego, maksymalnie pokrywając ich czas pracy zajęciami,
	\end{itemize}
	\item Wdrożenie systemu do rejestracji pacjentów,
	\begin{itemize}
		\item 50\% zmniejszenie kosztów obsługi rejestracji pacjentów, poprzez automatyzację tej procedury, co umożliwi redukcję etatów pracowników recepcji.
	\end{itemize}
\end{itemize}

\subsubsection{Użytkownicy produktu}

\begin{itemize}
	\item Klient
	\begin{itemize}
		\item Obsługa zakładu rehabilitacyjnego, administrująca czasem pracy specjalistów jak i~dostępem do sal.
	\end{itemize}
	\item Użytkownicy
	\begin{itemize}
			\item Wszyscy potencjalni klienci rejestrujący się na zajęcia rehabilitacyjne.
	\end{itemize}
\end{itemize}

\subsection{Ograniczenia projektu}

\subsubsection{Ograniczenia wynikające z natury i~okoliczności projektu}

\begin{itemize}
	\item Wszystkie dane klientów muszą podlegać ochronie zgodnie z ustawą o~ochronie danych osobowych (Dz.U. 1997 Nr 133 poz. 883),
	\item Moduły aplikacji zajmujące się administracją oraz rejestracją pacjentów mają zostać wykonane w~formie aplikacji webowej,
	\item Moduł obsługiwany przez lekarzy, ma być zaimplementowany w~formie aplikacji mobilnej (ograniczenie kosztów wdrożenia sprzętu po przez wykorzystanie inteligentnych urządzeń mobilnych, które posiadają lekarze),
	\item Do działania aplikacji konieczne jest połączenie z internetem,
	\item Wszystkie prace projektowe powinny zostać zaplanowane tak, aby umożliwić uruchomienie i~pełne funkcjonowanie systemu do 1~grudnia~2017~roku,
	\item Brak funduszy, ponieważ jest to projekt studencki.
\end{itemize}

\subsubsection{Konwencje nazewnicze i~definicje}
\begin{itemize}
	%TODO
	\item \textit{Aplikacja webowa} - program obsługiwany za pomocą przeglądarki internetowej,
	\item \textit{Aplikacja mobilna} - program uruchamiany na urządzeniu mobilnym typu smartphone lub tablet,
	\item \textit{Konto administratora} - konto wykorzystywane do autoryzacji i~obsługi modułu aplikacji służącego do zarządzania system,
	\item \textit{Ukrycie dostępności} lub \textit{usunięcie} - w~przypadku usuwania zabiegu, sprzętu, lekarza, etc. z~systemu mamy na myśli wyłączenie jego dostępności w~systemie, widoczności na listach, w~wyszukiwarce. Mimo to rekord zostaje dalej zapisany w~bazie danych, a~informacje o~wciąż są dostępne np. w~historii wizyt pacjent.
%	\item \textit{Logowanie} - proces autoryzacji użytkownika polegający podaniu identyfikatora i~hasła,
\end{itemize}

\subsubsection{Fakty i~założenia powiązane z~projektem}
\begin{itemize}
	\item Witryna ma być zaprojektowana w taki sposób, aby nie sprawiać trudności podczas rejestracji pacjentów,
	\item Ze względu na brak przewidzianego systemu rejestracji w~panelu administracyjnym konto administratora zostanie dodane do aplikacji na etapie produkcji, a~następnie klient będzie mógł zmienić w~nim dostępowe już przy pomocy funkcji wymienionych w~wymaganiach funkcjonalnych,
\end{itemize}

\subsection{Wymagania funkcjonalne}

\subsubsection{Moduł administracyjny}

\funcreq{
	1
}{
	Logowanie do panelu administracyjnego
}{
	System ma posiadać funkcjonalność logowania do panelu administracyjnego, przy pomocy konta administratora.
}{
	Uniemożliwienie dostępu osobą niepowołanym.
}{
	Użytkownik nie jest zalogowany do panelu administracyjnego.\\
	Weryfikacja użytkownika powinna zapewnić poufność i~bezpieczeństwo procesu.
}{
	Email oraz hasło
}{
	Użytkownik
}{
	Potwierdzenie tożsamości użytkownika i~przyznanie dostępu do panelu administracyjnego.
}{
	Wysoki
}{
	Funkcja została zaimplementowana zgodnie z~założeniami i~działa poprawnie.
}

\funcreq{
	2
}{
	Resetowanie hasła do konta administratora
}{
	System umożliwia zmianę hasła do konta administratora. Podanie w~formularz poprawnego adresu email przypisanego do tego konta, ma skutkować wysłaniem na wskazany adres wiadomości z~linkiem. Odnośnik ten ma prowadzić na podstronę ustawienia nowego hasła.
}{
	W~razie utraty hasła przez administratora systemu, ma umożliwić odzyskanie dostępu do panelu administracyjnego.
}{
	Użytkownik nie jest zalogowany do panelu administracyjnego.\\
	Bezpieczeństwo procesu.
}{
	Email
}{
	Użytkownik
}{
	Zmiana hasła do konta administratora.
}{
	Wysoki
}{
	Funkcja została zaimplementowana zgodnie z~założeniami i~działa poprawnie.
}

\funcreq{
	3
}{
	Zmiana hasła
}{
	System ma umożliwić zmianę hasła do konta administracyjnego.\\
	Zmiana hasła ma zostać potwierdzona linkiem wysyłanym na adres email przypisany do konta.
}{
	Pozwala odebrać dostęp osobą niepowołanym, które weszły w~posiadanie danych do konta.
}{
	Użytkownik jest zalogowany do panelu administracyjnego.\\
	Bezpieczeństwo procesu.
}{
	Aktualne hasło, nowe hasło i~potwierdzenie nowego hasła.
}{
	Użytkownik
}{
	Zmiana hasła do konta administratora.
}{
	Wysoki
}{
	Funkcja została zaimplementowana zgodnie z~założeniami i~działa poprawnie.
}

\funcreq{
	4
}{
	Zmiana adresu email
}{
	System ma umożliwić zmianę adresu email do konta administracyjnego.
}{
	W razie utraty dostępu lub bezpieczeństwa przypisanego adresu email do konta pozwala na jego zmianę.
}{
	Użytkownik jest zalogowany do panelu administracyjnego.\\
	Bezpieczeństwo procesu.
}{
	Aktualne hasło i~email.
}{
	Użytkownik
}{
	Zmiana adresu email przypisanego do konta administratora.
}{
	Wysoki
}{
	Funkcja została zaimplementowana zgodnie z~założeniami i~działa poprawnie.
}

\funcreq{
	5
}{
	Wylogowanie z~panelu administracyjnego
}{
	System ma umożliwić wylogowanie z~panelu administracyjnego.
}{
	Ma to uniemożliwić dostęp do panelu osobą trzecim, które uzyskały dostęp do komputera z~którego wykonywane są logowania do niego.
}{
	Użytkownik jest zalogowany do panelu administracyjnego.\\
	Bezpieczeństwo procesu.
}{
	-
}{
	Użytkownik
}{
	Odebranie dostęp do funkcjonalności panelu administracyjnego.
}{
	Wysoki
}{
	Funkcja została zaimplementowana zgodnie z~założeniami i~działa poprawnie.
}

\funcreq{
	6
}{
	Dodanie sprzętu
}{
	System umożliwia dodanie nowego sprzętu rehabilitacyjnego do bazy danych.\\
}{
	Aby móc administrować zasobami konieczne jest ich dodanie do bazy danych w~systemie.
}{
	Użytkownik jest zalogowany do panelu administracyjnego.\\
	Nie może być dwóch sprzętów o~tej samej nazwie.\\
	Ilość musi być większa od zera.
}{
	Nazwa sprzętu i~ilość.
}{
	Użytkownik
}{
	Dodanie nowego sprzętu do bazy danych.
}{
	Wysoki
}{
	Funkcja została zaimplementowana zgodnie z~założeniami i~działa poprawnie.
}

\funcreq{
	7
}{
	Lista sprzętu
}{
	System umożliwia przeglądanie oraz filtorwanie sprzętu po nazwie, ilości, nazwie zabiegu, imieniu~i nazwisku lekarza.
}{
	System musi umożliwiać przeglądanie dostępnego w~bazie danych sprzętu.
}{
	Użytkownik jest zalogowany do panelu administracyjnego.
}{
	Nazwa sprzętu (opcjonalne), ilość(opcjonlane), nazwa zabiegu (opcjonalne), imie i~naziwsko lekarza.
}{
	Użytkownik
}{
	Wyświetlenie listy sprzętów, które spełnia kryteria z~filtrów.
}{
	Wysoki
}{
	Funkcja została zaimplementowana zgodnie z~założeniami i~działa poprawnie.
}

\funcreq{
	8
}{
	Podgląd sprzętu
}{
	System umożliwia podgląd sprzętu.
}{
	System musi umożliwiać podgląd nazwy, ilości sprzętu.\\
	System ma także wyświetlić listę zabiegów wykorzystujących ten sprzęt.\\
	System ma także wyświetlić listę lekarzy wykorzystujących ten sprzęt.
}{
	Użytkownik jest zalogowany do panelu administracyjnego.
}{
	-
}{
	Użytkownik
}{
	Wyświetlenie opis sprzętów oraz listę powiązanych z~nim zabiegów i~lekarzy.
}{
	Wysoki
}{
	Funkcja została zaimplementowana zgodnie z~założeniami i~działa poprawnie.
}

\funcreq{
	9
}{
	Edycja sprzętu
}{
	System umożliwia edycję informacji o~sprzęcie.
}{
	System ma umożliwić zmianę błędnej nazwy lub ilości sprzętu.
}{
	Użytkownik jest zalogowany do panelu administracyjnego.\\
	Nie może być dwóch sprzętów o~tej samej nazwie.\\
	Ilość musi być większa od zera.
}{
	Nazwa sprzętu i~ilość.
}{
	Użytkownik
}{
	Zmiana nazwy i~ilości sprzętu.
}{
	Wysoki
}{
	Funkcja została zaimplementowana zgodnie z założeniami i działa poprawnie.
}

\funcreq{
	10
}{
	Usuwanie sprzętu
}{
	System ma umożliwić usuwanie sprzętu z~bazy danych.\\
	Usuwanie ma się odbywać po przez nadanie rekordowi w~bazie danych statusu usuniętego, ukrywając jego dostępność w~całym systemie, ale nie usuwając całkowicie informacji o~nim z~bazy danych.\\
	Podczas próby usunięcia sprzętu do którego jest przypisany co najmniej jeden zabieg musi zostać wyświetlony stosowny komunikat.
}{
	W~razie zniszczenia, bądź rezygnacji z~danego sprzętu w~zakładzie musi istnieć możliwość usunięcia.
}{
	Użytkownik jest zalogowany do panelu administracyjnego.\\
	Do sprzętu nie może być przypisany żaden zabieg.
}{
	Sprzęt
}{
	Użytkownik
}{
	Ukrycie dostępności sprzętu w~całym systemie na zawsze.
}{
	Wysoki
}{
	Funkcja została zaimplementowana zgodnie z założeniami i działa poprawnie.
}

\funcreq{
	11
}{
	Dodanie zabiegu
}{
	System umożliwia dodanie nowego zabiegu rehabilitacyjnego do bazy danych.\\
}{
	Aby móc administrować zasobami konieczne jest ich dodanie do bazy danych w~systemie.
}{
	Użytkownik jest zalogowany do panelu administracyjnego.\\
	Nie może być dwóch zabiegów o~tej samej nazwie.
}{
	Nazwa zabiegu, czas trwania (wybór opcji: 30m, 1h, 1.5h, 2h), cena i~wymagane sprzęty (opcjonalne).
}{
	Użytkownik
}{
	Dodanie nowego zabiegu do bazy danych.
}{
	Wysoki
}{
	Funkcja została zaimplementowana zgodnie z~założeniami i~działa poprawnie.
}

\funcreq{
	12
}{
	Lista zabiegów
}{
	System umożliwia przeglądanie oraz filtorwanie zabiegów po nazwie, czasie trwania, cenie, nazwie sprzętu, imieniu i~nazwisku lekarza.
}{
	System musi umożliwiać przeglądanie dostępnych w~bazie danych zabiegów.
}{
	Użytkownik jest zalogowany do panelu administracyjnego.
}{
	Nazwa zabiegu (opcjonalne), czas trwania (wybór opcji: 30m, 1h, 1.5h, 2h), cena (opcjonalne), nazwa sprzętu (opcjonlane), imie i~naziwsko lekarza.
}{
	Użytkownik
}{
	Wyświetlenie listy zabiegów, które spełnia kryteria z~filtrów.
}{
	Wysoki
}{
	Funkcja została zaimplementowana zgodnie z~założeniami i~działa poprawnie.
}

\funcreq{
	13
}{
	Podgląd zabiegu
}{
	System umożliwia podgląd zabiegu.
}{
	System musi umożliwiać podgląd nazwy i ceny.\\
	System ma także wyświetlić listę sprzętów wykorzystywanych do tego zabiegu.\\
	System ma także wyświetlić listę lekarzy wykonujących ten zabieg.
}{
	Użytkownik jest zalogowany do panelu administracyjnego.
}{
	-
}{
	Użytkownik
}{
	Wyświetlenie opis zabiegu oraz listę powiązanych z~nim sprzętów i~lekarzy.
}{
	Wysoki
}{
	Funkcja została zaimplementowana zgodnie z~założeniami i~działa poprawnie.
}

\funcreq{
	14
}{
	Edycja zabiegu
}{
	System umożliwia edycję informacji o~zabiegu.
}{
	System ma umożliwić zmianę błędnej nazwy, czasu trwania, ceny lub powiązanego sprzętu.
}{
	Użytkownik jest zalogowany do panelu administracyjnego.\\
	Nie może być dwóch zabiegów o~tej samej nazwie.
}{
	Nazwa zabiegu, czas trwania (wybór opcji: 30m, 1h, 1.5h, 2h), cena i~wymagane sprzęty (opcjonalne).
}{
	Użytkownik
}{
	Zmiana nazwy lub przypisanego sprzętu do zabiegu w~bazy danych.
}{
	Wysoki
}{
	Funkcja została zaimplementowana zgodnie z~założeniami i~działa poprawnie.
}

\funcreq{
	15
}{
	Usuwanie zabiegu
}{
	System ma umożliwić usuwanie zabiegu z~bazy danych.\\
	Usuwanie ma się odbywać po przez nadanie rekordowi w~bazie danych statusu usuniętego, ukrywając jego dostępność w~całym systemie, ale nie usuwając całkowicie informacji o~nim z~bazy danych.\\
	Podczas próby usunięcia zabiegu do którego jest przypisany co najmniej jeden lekarz musi zostać wyświetlony stosowny komunikat.
}{
	W~razie rezygnacji z~wykonywanie danego zabiegu w~zakładzie musi istnieć możliwość usunięcia go z~systemu.
}{
	Użytkownik jest zalogowany do panelu administracyjnego.\\
	Do zabiegu nie może być przypisany żaden lekarz.
}{
	Zabieg
}{
	Użytkownik
}{
	Ukrycie dostępności zabiegu w~całym systemie na zawsze.
}{
	Wysoki
}{
	Funkcja została zaimplementowana zgodnie z założeniami i działa poprawnie.
}

\funcreq{
	16
}{
	Dodanie lekarza
}{
	System umożliwia dodanie nowego lekarza do bazy danych.\\
	Po dodaniu lekarza do bazy danych, na wpisany w~formularzu adres email wysyłany jest link, który pozwoli na ustawinie hasła do konta.
}{
	Aby móc administrować zasobami konieczne jest ich dodanie do bazy danych w~systemie.
}{
	Użytkownik jest zalogowany do panelu administracyjnego.\\
	Nie może być dwóch lekarzy z~tym samym adresem email.
}{
	Imię lekarza, nazwisko, email, specjalizacja lekarza, opis, zdjęcie (opcjonlane), lista wykonywanych zabiegów (conajmniej jeden).
}{
	Użytkownik
}{
	Dodanie nowego zabiegu do bazy danych.
}{
	Wysoki
}{
	Funkcja została zaimplementowana zgodnie z~założeniami i~działa poprawnie.
}

\funcreq{
	17
}{
	Lista lekarzy
}{
	System umożliwia przeglądanie oraz filtorwanie lekarzy po imieniu, nazwisku i wykonywanych zabiegach.
}{
	System musi umożliwiać przeglądanie dostępnych w~bazie danych lekarzy.
}{
	Użytkownik jest zalogowany do panelu administracyjnego.
}{
	Imię (opcjonalne), nazwisko (opcjonalne), email (opcjonalne), nazwa zabiegu(opcjonalne) i sprzęt(opcjonalne).
}{
	Użytkownik
}{
	Wyświetlenie listy lekarzy, którzy spełniają kryteria z~filtrów.
}{
	Wysoki
}{
	Funkcja została zaimplementowana zgodnie z~założeniami i~działa poprawnie.
}

\funcreq{
	18
}{
	Podgląd lekarza
}{
	System umożliwia podgląd lekarza.
}{
	System musi umożliwiać podgląd lekarza.\\
	System ma także wyświetlić listę sprzętów wykorzystywanych przez lekarza.\\
	System ma także wyświetlić listę zabiegów wykonywanych przez lekarza.
}{
	Użytkownik jest zalogowany do panelu administracyjnego.
}{
	-
}{
	Użytkownik
}{
	Wyświetlenie opis lekarza oraz listy powiązanych z~nim sprzętów i~zabiegów.
}{
	Wysoki
}{
	Funkcja została zaimplementowana zgodnie z~założeniami i~działa poprawnie.
}

\funcreq{
	19
}{
	Edycja lekarza
}{
	System umożliwia edycję informacji o~lekarzu.
}{
	System ma umożliwić zmianę błędnych danych lub powiązanych zabiegów.
}{
	Użytkownik jest zalogowany do panelu administracyjnego.\\
	Nie może być dwóch lekarzy z~tym samym adresem email.
}{
	Imię lekarza, nazwisko, email, specjalizacja lekarza, opis, zdjęcie (opcjonlane), lista wykonywanych zabiegów (conajmniej jeden).
}{
	Użytkownik
}{
	Zmiana danych lekarza w~bazy danych.
}{
	Wysoki
}{
	Funkcja została zaimplementowana zgodnie z~założeniami i~działa poprawnie.
}

\funcreq{
	20
}{
	Zmiana hasła lekarza
}{
	System umożliwia wysłanie odnośnika na adres email przypisany do konta lekarza umożliwiając zmianę hasła.\\
	Po otwarciu linku pojawia się formularz do wprowadzenia nowego hasła.
}{
	Pozwala odebrać dostęp osobą niepowołanym, które weszły w~posiadanie danych do konta.
}{
	Dla panelu administracyjnego: Użytkownik jest zalogowany do panelu administracyjnego.
	Dla linku zmiany hasła: Użytkownik jest nie zalogowany do panelu administracyjnego.
}{
	Lekarz | Nowe hasło i potwierdzenie nowego hasła.
}{
	Użytkownik
}{
	Zmiana hasła do konta lekarza w~bazy danych.
}{
	Wysoki
}{
	Funkcja została zaimplementowana zgodnie z~założeniami i~działa poprawnie.
}

\funcreq{
	21
}{
	Usuwanie lekarza
}{
	System ma umożliwić usuwanie lekarza z~bazy danych.\\
	Usuwanie ma się odbywać po przez nadanie rekordowi w~bazie danych statusu usuniętego, ukrywając jego dostępność w~całym systemie, ale nie usuwając całkowicie informacji o~nim z~bazy danych.\\
	Podczas próby usunięcia lekarza do którego jest umówiona co najmniej jedna wizyta musi zostać wyświetlony stosowny komunikat.
}{
	W~razie zakończenia pracy lekarza w~zakładzie musi istnieć możliwość usunięcia go z~listy dostępnych lekarzy.
}{
	Użytkownik jest zalogowany do panelu administracyjnego.\\
	Lekarz nie ma umówionych wizyt.
}{
	Lekarz
}{
	Użytkownik
}{
	Ukrycie dostępności lekarza w~całym systemie na zawsze.
}{
	Wysoki
}{
	Funkcja została zaimplementowana zgodnie z założeniami i działa poprawnie.
}

\funcreq{
	22
}{
	Dodanie terminów wizyt lekarzowi
}{
	System ma umożliwić dodanie terminów wizyt lekarzowi.\\
}{
	Dodanie terminów przyjęć na zabiegi będzie źródłem danych o~godzinach zabiegów dla pacjentów oraz czasie zajętości sprzętu.
}{
	Użytkownik jest zalogowany do panelu administracyjnego.\\
	Lekarz nie ma w tym czasie innych terminów zabiegów.\\
	Sprzęt potrzebny do wykonania zabiegów jest dostępny.
}{
	Lekarz, zabieg, dni tygodnia i przedział godzin w~ciągu dnia w~których odbywa się zabieg u tego lekarza i przedział dat dla których mają zostać dodane terminy zabiegów.
}{
	Użytkownik
}{
	Dodanie terminów zabiegów do bazy danych.
}{
	Wysoki
}{
	Funkcja została zaimplementowana zgodnie z założeniami i działa poprawnie.
}

\funcreq{
	23
}{
	Wyświetlanie kalendarza
}{
	System ma umożliwić wyświetlenie kalendarza wizyt dla danego lekarza.
}{
	Musi istnieć podgląd kalendarza wizyt, aby mieć wgląd w grafik pracy lekarza.
}{
	Użytkownik jest zalogowany do panelu administracyjnego.
}{
	Lekarz
}{
	Użytkownik
}{
	Wyświetlenie kalendarza lekarza.
}{
	Wysoki
}{
	Funkcja została zaimplementowana zgodnie z założeniami i działa poprawnie.
}

\funcreq{
	24
}{
	Usunięcie terminu
}{
	System ma umożliwić usunięcie dostępności terminu.\\
	W razie gdy na dany termin jest umówiony pacjent ma pojawić się stosowny komunikat.
}{
	Potrzebne jest narzędzie, które pozwoli odowałać terminy zabiegów.
}{
	Użytkownik jest zalogowany do panelu administracyjnego.\\
	Na dany termin nie może być umówiony pacjent.\\
	Termin jeszcze nie upłynął.
}{
	Termin
}{
	Użytkownik
}{
	Usunięcie terminu z kalendarza.
}{
	Wysoki
}{
	Funkcja została zaimplementowana zgodnie z założeniami i działa poprawnie.
}

\funcreq{
	25
}{
	Odwołanie terminu
}{
	System ma umożliwić odwołanie wizyty.\\
	Pacjent ma zostać powiadomiony o~odwołaniu jego wizyty mailem.
}{
	Musi istnieć narzędzie pozwalające na odwołanie wizyt w~razie braku możliwości ich realizacji.
}{
	Użytkownik jest zalogowany do panelu administracyjnego.\\
	Na dany termin jest zapisany pacjent.\\
	Termin jeszcze nie upłynął.
}{
	Termin
}{
	Użytkownik
}{
	Odwołanie umówionego zabiegu.
}{
	Wysoki
}{
	Funkcja została zaimplementowana zgodnie z założeniami i działa poprawnie.
}

\funcreq{
	26
}{
	Podsumowanie terminu
}{
	Po odbyciu się wizyty na umówiony termin, ma istnieć możliwość jej potwierdzenia, że się odbyła oraz dodania krótkiej opcjonalnej informacji o jej przebiegu.
}{
	Musi istnieć miejsce do gromadzenia historii przebiegu zabiegów.
}{
	Użytkownik jest zalogowany do panelu administracyjnego.\\
	Na dany termin jest zapisany pacjent.\\
	Termin upłynął.
}{
	Termin, potwierdzenie/zaprzecznie odbycia zabiegu i opis (opcjonalne).
}{
	Użytkownik
}{
	Dodanie informacji podsumowującej miony umówiony zabieg.
}{
	Wysoki
}{
	Funkcja została zaimplementowana zgodnie z założeniami i działa poprawnie.
}

\subsubsection{Moduł pacjenta}

\funcreq{
	27
}{
	Rejestracja pacjenta
}{
	System ma posiadać funkcjonalność rejestracji dla pacjenta.\\
	Rejestracja ma zostać potwierdzona po przez otworzenie przez użytkownika linku, który zostanie wysłany na jego skrzynkę mailową po wysłaniu poprawnie wypełnionego formularza rejestracyjnego.
}{
	System ma posiadać funkcjonalność rejestracji dla pacjenta. Co umożliwi utworzenie konta do którego zostaną przypisane podstawowe informacje, a także dzięki, któremu użytkownik będzie miał dostęp do umówionych wizyt i~ich historii.
}{
	Użytkownik nie jest zalogowany.\\
	Nie może być dwóch użytkowników z~tym samym adresem email.
}{
	Imię, nazwisko, email, numer telefonu, hasło i powtórz hasło.
}{
	Użytkownik
}{
	Dodanie konta pacjenta do bazy danych.
}{
	Wysoki
}{
	Funkcja została zaimplementowana zgodnie z~założeniami i~działa poprawnie.
}

\funcreq{
	28
}{
	Logowanie pacjenta
}{
	System ma posiadać funkcjonalność logowania dla pacjenta.
}{
	Umożliwia dostęp do rejestracji i~prywatnej historii pacjentowi.
}{
	Użytkownik nie jest zalogowany do panelu administracyjnego.\\
	Weryfikacja użytkownika powinna zapewnić poufność i~bezpieczeństwo procesu.
}{
	Email oraz hasło
}{
	Użytkownik
}{
	Potwierdzenie tożsamości użytkownika i~przyznanie dostępu do opcji dostępnych po zalogowaniu w~module dla pacjenta.
}{
	Wysoki
}{
	Funkcja została zaimplementowana zgodnie z~założeniami i~działa poprawnie.
}

\funcreq{
	29
}{
	Resetowanie hasła do konta pacjenta
}{
	System umożliwia zmianę hasła do konta pacjenta. Podanie w~formularz poprawnego adresu email przypisanego do tego konta, ma skutkować wysłaniem na wskazany adres wiadomości z~linkiem. Odnośnik ten ma prowadzić na podstronę ustawienia nowego hasła.
}{
	W~razie utraty hasła przez pacjenta, ma umożliwić odzyskanie dostępu do konta.
}{
	Użytkownik nie jest zalogowany do panelu administracyjnego.\\
	Bezpieczeństwo procesu.
}{
	Email
}{
	Użytkownik
}{
	Zmiana hasła do konta pacjenta.
}{
	Wysoki
}{
	Funkcja została zaimplementowana zgodnie z~założeniami i~działa poprawnie.
}

\funcreq{
	30
}{
	Zmiana hasła
}{
	System ma umożliwić zmianę hasła do konta pacjenta.\\
	Zmiana hasła ma zostać potwierdzona linkiem wysyłanym na adres email przypisany do konta.
}{
	Pozwala odebrać dostęp osobą niepowołanym, które weszły w~posiadanie danych do konta.
}{
	Użytkownik jest zalogowany do panelu administracyjnego.\\
	Bezpieczeństwo procesu.
}{
	Aktualne hasło, nowe hasło i~potwierdzenie nowego hasła.
}{
	Użytkownik
}{
	Zmiana hasła do konta pacjenta.
}{
	Wysoki
}{
	Funkcja została zaimplementowana zgodnie z~założeniami i~działa poprawnie.
}

\funcreq{
	31
}{
	Zmiana adresu email
}{
	System ma umożliwić zmianę adresu email do konta pacjenta.
}{
	W razie utraty dostępu lub bezpieczeństwa przypisanego adresu email do konta pozwala na jego zmianę.
}{
	Użytkownik jest zalogowany do panelu administracyjnego.\\
	Bezpieczeństwo procesu.
}{
	Aktualne hasło i~email.
}{
	Użytkownik
}{
	Zmiana adresu email przypisanego do konta pacjenta.
}{
	Wysoki
}{
	Funkcja została zaimplementowana zgodnie z~założeniami i~działa poprawnie.
}

\funcreq{
	32
}{
	Wylogowanie z~modułu pacjenta
}{
	System ma umożliwić wylogowanie z~modułu pacjenta.
}{
	Ma to uniemożliwić dostęp do poufnych danych osobą trzecim, które uzyskały dostęp do komputera z~którego wykonywane są logowania.
}{
	Użytkownik jest zalogowany do modułu pacjenta.\\
	Bezpieczeństwo procesu.
}{
	-
}{
	Użytkownik
}{
	Odebranie dostęp do funkcjonalności po zalogowaniu panelu pacjenta.
}{
	Wysoki
}{
	Funkcja została zaimplementowana zgodnie z~założeniami i~działa poprawnie.
}

\funcreq{
	33
}{
	Wyszukiwanie
}{
	System ma umożliwić wyszukiwanie zabiegów i~lekarzy.\\
	Wyszukiwanie ma się odbywać przez wprowadzenie frazy pasującej do nazwy zabiegu lub specjalizacji lekarza lub imię i~nazwiska lekarza.\\
	W~wynikach ma wyświetlić listę lekarzy pasujących do wprowadzonych danych.\\
	Wybranie danego lekarza ma przenieść do jego publicznego profilu.
}{
	Ma to umożliwić znalezienie właściwego lekarza. 
}{
	-
}{
	Szukana fraza
}{
	Użytkownik
}{
	Wyświetlenie zgodnych wyników.
}{
	Wysoki
}{
	Funkcja została zaimplementowana zgodnie z~założeniami i~działa poprawnie.
}

\funcreq{
	34
}{
	Profil lekarza
}{
	Wyświetla opis lekarza na podstawie danych wprowadzonych w~panelu administracyjnym.\\
	Wyświetla kalendarz z~terminami wizyt lekarza.\\
	Przy terminach wizyt mają pojawiać się nazwy wykonywanych w~tedy zabiegów i~ich ceny.\\
	Ma pojawiać się informacja czy termin jest przeszły, dostępny lub zajęty.\\
	Dostępne terminy mają być aktywne do kliknięcia.
}{
	Ma to umożliwić sprawdzenie lekarza przez pacjenta i~przeglądnięcie jego terminarza, a także w razie chęci skorzystania z~zabiegu przejście dalej.
}{
	Użytkownik jest zalogowany do modułu pacjenta.
}{
	Lekarz
}{
	Użytkownik
}{
	Wyświetlenie opisu lekarza z~~jego kalendarzem.
}{
	Wysoki
}{
	Funkcja została zaimplementowana zgodnie z~założeniami i~działa poprawnie.
}

\funcreq{
	35
}{
	Umówienie na wizytę
}{
	Kliknięcie w~odnośnik w~kalendarzu z~wymagania 31, w przypadku niezalogowanego użytkownika przenosi do strony logowania.\\
	W~przypadku zalogowanego wyświetla pytanie z potwierdzeniem wizyty.
}{
	Ma to umożliwić rejestrację na wizytę.
}{
	Użytkownik jest zalogowany do modułu pacjenta.\\
	Dany termin musi być w~przyszłości, nie wcześniej niż za 3h.\\
	Dany termin musi być wolny.
}{
	Termin
}{
	Użytkownik
}{
	Zapisanie na termin.
}{
	Wysoki
}{
	Funkcja została zaimplementowana zgodnie z~założeniami i~działa poprawnie.
}

\funcreq{
	36
}{
	Wizyty
}{
	Użytkownik ma możliwość przeglądnięcia historii wizyt oraz wyświetlenia ich opisu.\\
	W historii mają być dostępne również przyszłe wizyty.
}{
	Ma to umożliwić wgląd w historię wizyt.
}{
	Użytkownik jest zalogowany do modułu pacjenta.
}{
	Szukana fraza
}{
	Użytkownik
}{
	Wyświetlenie historii wizyt.
}{
	Wysoki
}{
	Funkcja została zaimplementowana zgodnie z~założeniami i~działa poprawnie.
}

\funcreq{
	37
}{
	Odwołanie wizyty
}{
	Użytkownik ma możliwość odwołania wizyty.
}{
	Jeśli pacjent nie będzie mógł pojawić się na wizycie powinien móc z niej zrezygnować.
}{
	Użytkownik jest zalogowany do modułu pacjenta.\\
	Wizyta z przyszłości, nie wcześniej niż za 3h.
}{
	Wizyta.
}{
	Użytkownik
}{
	Odwołanie wizyty.
}{
	Wysoki
}{
	Funkcja została zaimplementowana zgodnie z~założeniami i~działa poprawnie.
}

\subsubsection{Moduł mobilny dla lekarza}

\funcreq{
	38
}{
	Logowanie lekarza
}{
	System ma posiadać funkcjonalność logowania dla lekarza.
}{
	Umożliwia dostęp do kalendarza lekarza i~wprowadzania w~nim zmian.
}{
	Użytkownik nie jest zalogowany do aplikacji lekarza.\\
	Weryfikacja użytkownika powinna zapewnić poufność i~bezpieczeństwo procesu.
}{
	Email oraz hasło
}{
	Użytkownik
}{
	Potwierdzenie tożsamości użytkownika i~przyznanie dostępu do opcji dostępnych po zalogowaniu w~aplikacji dla lekarza.
}{
	Wysoki
}{
	Funkcja została zaimplementowana zgodnie z~założeniami i~działa poprawnie.
}

\funcreq{
	39
}{
	Wylogowanie z~aplikacji lekarza
}{
	System ma umożliwić wylogowanie z~aplikacji lekarza.
}{
	Ma to uniemożliwić dostęp do poufnych danych osobą trzecim, które uzyskały dostęp do komputera z~którego wykonywane są logowania.
}{
	Użytkownik jest zalogowany do aplikacji lekarza.\\
	Bezpieczeństwo procesu.
}{
	-
}{
	Użytkownik
}{
	Odebranie dostęp do funkcjonalności po zalogowaniu do aplikacji lekarza.
}{
	Wysoki
}{
	Funkcja została zaimplementowana zgodnie z~założeniami i~działa poprawnie.
}

\funcreq{
	40
}{
	Wyświetlenie kalendarza
}{

	Wyświetla kalendarz z~terminami wizyt lekarza.\\
	Przy terminach wizyt mają pojawiać się nazwy wykonywanych w~tedy zabiegów.\\
	Ma pojawiać się informacja czy termin jest przeszły, dostępny lub zajęty.\\
	Dostępne i aktywne terminy mają być aktywne do kliknięcia.
}{
	Ma to umożliwić sprawdzenie lekarzowi swojego planu zajęć.
}{
	Użytkownik jest zalogowany do aplikacji lekarza.
}{
	-
}{
	Użytkownik
}{
	Wyświetlenie kalendarza.
}{
	Wysoki
}{
	Funkcja została zaimplementowana zgodnie z~założeniami i~działa poprawnie.
}

\funcreq{
	41
}{
	Dodanie terminów wizyt przez lekarza
}{
	System ma umożliwić dodanie terminów wizyt przez apliakcję lekarzowi.\\
}{
	Dodanie terminów przyjęć na zabiegi będzie źródłem danych o~godzinach zabiegów dla pacjentów oraz czasie zajętości sprzętu.
}{
	Użytkownik jest zalogowany do aplikacji lekarza.\\
	Lekarz nie ma w tym czasie innych terminów zabiegów.\\
	Sprzęt potrzebny do wykonania zabiegów jest dostępny.
}{
	Zabieg, dni tygodnia i przedział godzin w~ciągu dnia w~których odbywa się zabieg u tego lekarza i przedział dat dla których mają zostać dodane terminy zabiegów.
}{
	Użytkownik
}{
	Dodanie terminów zabiegów do bazy danych.
}{
	Wysoki
}{
	Funkcja została zaimplementowana zgodnie z założeniami i działa poprawnie.
}

\funcreq{
	42
}{
	Odwołanie wizyty
}{
	Lekarz ma możliwość odwołania wizyty.\\
	W momencie odwołania wizyty pacjent dostaje wiadomość email z informacją.
	
	%TODO HERE
	
}{
	Jeśli pacjent nie będzie mógł pojawić się na wizycie powinien móc z niej zrezygnować.
}{
	Użytkownik jest zalogowany do modułu pacjenta.\\
	Wizyta z przyszłości, nie wcześniej niż za 3h.
}{
	Wizyta.
}{
	Użytkownik
}{
	Odwołanie wizyty.
}{
	Wysoki
}{
	Funkcja została zaimplementowana zgodnie z~założeniami i~działa poprawnie.
}


\subsubsection{Przypadki użycia}

\includegraphics[scale=0.4]{graphs/przypadki_uzycia.jpg}


%\include{chapters/wstep}
%\include{chapters/projekt}
%\include{chapters/opis_implementacji}
%\include{chapters/bibliografia}
%\include{chapters/inwentaryzacha_sprzetu_i_infrastuktury_dostepnej_w_przedsiborstwie}
%\include{chapters/analiza_potrzeb_uzytkownika}

\end{document}
