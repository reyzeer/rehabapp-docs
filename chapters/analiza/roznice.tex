\subsection{Różnice pomiędzy implementacją a specyfikacją wymagań}


\begin{itemize}
	\item Aplikacja w języku angielskim
	\begin{itemize}
		\item Językiem interfejsu użytkownika jest język angielski. W niesprecyzowanych założeniach początkowych (język aplikacji nie został określony w wymaganiach pozafunkcjonalnych) zakładany był język polski, jednak ze względu na możliwość ewentualnej rozbudowy, przyjęta została nomenklatura angielska.
	\end{itemize}
	\item Moduł kalendarza
	\begin{itemize}
		\item System miał umożliwiać wyświetlanie kalendarza wizyt dla danego lekarza - aby każdy użytkownik systemu mógł przeglądać harmonogram jego pracy. W trakcie prac została przyjęta koncepcja stworzenia listy wraz z dostępnymi terminami wizyt. Zmiana była spowodowana trudnością implementacyjną pierwotnego założenia.
	\end{itemize}
	
		\item Rejestracja pacjentów i historia pacjenta
	\begin{itemize}
		\item System miał umożliwiać rejestrację pacjenta na wizyty - z powodów terminowych funkcjonalność nie została ukończona i udostępniona. Co za tym idzie - użytkownik nie ma możliwości wglądu do swojej historii.
	\end{itemize}
	


\end{itemize}

