\subsection{Odchylenia w realizacji planowanych zadań}

\begin{itemize}

	\item Spotkania grupy projektowej
	\begin{itemize}
		\item Z założenia, grupa projektowa miała się spotykać cyklicznie i razem opracowywać powierzone zadania. Z powodu rozbieżności w planach zajęć, trudno było wyznaczyć termin, który pasowałby każdemu członkowi zespołu.
	\end{itemize}
	\item Plan komunikacji
		\begin{itemize}
		\item Początkowy plan komunikacji zakładał największy przepływ informacji przez portale społecznościowe, rozmowy telefoniczne. Z biegiem czasu, okazało się, że najefektywniejszą formą przekazywania założeń oraz analizy zadań, są spotkania grupowe.
	\end{itemize}
	
		\item Podział modułowy
		\begin{itemize}
		\item W zespole funkcjonowały podgrupy - każda z nich miała określone zadania do wykonania, opisane w dokumentacji. Z powodu braku wystarczającej znajomości niektórych technologii przez niektórych członków grupy, część zadań była realizowana “płynnie” między zespołami.
	\end{itemize}
	
		\item Realizacja funkcjonalności
		\begin{itemize}
		\item Z racji tego, iż niektóre rozwiązania funkcjonalne były bliźniaczo podobne (na przykład w module pacjenta jak i administratora), część z nich, po modyfikacjach, była wykorzystywana w innych modułach.
	\end{itemize}
	
\end{itemize}

