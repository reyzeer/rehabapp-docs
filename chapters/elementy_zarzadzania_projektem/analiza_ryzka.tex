\subsection{Analiza ryzyka}

\risk{
	Przekroczenie planowanego terminu oddania projektu.
}{
	Obniżona ocena bądź brak zaliczenia z przedmiotu PTSIM.
}{
	Duże.
}{
	Wysokie.
}{
	Cykliczne spotkania zespołu projektowego, omawianie postępów i problemów, określenie ram czasowych na wykonanie poszczególnych zadań projektowych.
}

\risk{
	Awarie techniczne sprzętu.
}{
	Brak możliwości tworzenia/rozwijania oprogramowania/utrata danych.
}{
	Średnie.
}{
	Średnie.
}{
	Konserwacja i odpowiednie zabezpieczanie sprzętu.
}

\risk{
	Problemy z komunikacją w zespole.
}{
	Członkowie grupy nie wiedzą, czym powinni się zająć.
}{
	Małe.
}{
	Średnie.
}{
	Stworzenie wielu kanałów komunikacji (internet, telefon, spotkania), kontrola każdego członka zespołu przez lidera.
}

\risk{
	Brak umiejętności pracy w grupie przez członków zespołu. Brak doświadczenia w pracy grupowej u członków zespołu
}{
	Opóźnienia w tworzeniu modułów, tworzenie się podgrup wśród członków zespołu.
	Opóźnienia w tworzeniu modułów, tworzenie się podgrup wśród członków zespołu mających inny pomysł wykonania projektu czego wynikiem może być rozłam grupy
	
}{
	Małe.
}{
	Małe.
}{
	Tworzenie przyjaznej atmosfery, organizowanie wyjść integracyjnych.
}

\risk{
	Niedoświadczony zespół projektowy. 
}{
	Opóźnienia, słaba jakość kodu, produkt nie spełnia założonych wymogów.
}{
	Duże.
}{
	Wysokie.
}{
	Doszkalanie grupy z nieznanych im technologii.
}

\risk{
	Niedyspozycyjność członka grupy projektowej.
}{
	Trudność w podziale zadań.
}{
	Duże.
}{
	Średnie.
}{
	Niedyspozycyjnosć jest najczęściej spowodowana zdarzeniem czysto losowym - ciężko ją przewidzieć jak i jej przeciwdziałać.
}

\risk{
	Utrata fragmentów kodu.
}{
	Konieczność tworzenia rozwiązań od nowa.
}{
	Małe.
}{
	Średnie.
}{
	Kod przechowywany w wielu miejscach, wykorzystanie repozytorium.
}

\risk{
	Zdefiniowanie niepotrzebnych wymagań.
}{
	Podrzędne cele zabierają za dużo czasu, którego może zabraknąć do osiągnięcia sukcesu.
}{
	Średnie.
}{
	Małe.
}{
	Jasno określony czas pracy nad konkretnymi modułami.
}

\risk{
	Zmiana wymagań klienta w trakcie tworzenia oprogramowania.
}{
	Redefinicja projektu, możliwość jego upadku.
}{
}{
	Wysokie.
}{
	Wyznaczenie stałych terminów na wprowadzanie ewentualnych poprawek bądź nowych funkcjonalności.
}

\risk{
	Wycofanie się klienta z projektu.
}{
	Upadek projektu.
}{
	Małe.
}{
	Wysokie.
}{
	Stały kontakt z klientem, umowa nakładająca karę na klienta w wypadku wycofania się z projektu w celu zapewnienia funduszy zapobiegających upadkowi firmy.
}
