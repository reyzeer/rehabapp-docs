\section{Specyfikacja wymagań}

\subsection{Czynniki sterujące projektem}

\subsubsection{Cele projektu}

\begin{itemize}
	\item Wdrożenie aplikacji do zarządzania dniem pracy specjalistów
	i~przypisywania im sprzętu rehabilitacyjnego na określoną porę,
	\begin{itemize}
		\item 30\% zwiększenie efektywności pracy poprzez usprawnienie systemu przypisywania specjalistom dostępu do sprzętu rehabilitacyjnego, maksymalnie pokrywając ich czas pracy zajęciami,
	\end{itemize}
	\item Wdrożenie systemu do rejestracji pacjentów,
	\begin{itemize}
		\item 50\% zmniejszenie kosztów obsługi rejestracji pacjentów, poprzez automatyzację tej procedury, co umożliwi redukcję etatów pracowników recepcji.
	\end{itemize}
\end{itemize}

\subsubsection{Użytkownicy produktu}

\begin{itemize}
	\item Klient
	\begin{itemize}
		\item Obsługa zakładu rehabilitacyjnego, administrująca czasem pracy specjalistów jak i~dostępem do sal.
	\end{itemize}
	\item Użytkownicy
	\begin{itemize}
			\item Wszyscy potencjalni klienci rejestrujący się na zajęcia rehabilitacyjne.
	\end{itemize}
\end{itemize}


\subsection{Czynniki sterujące projektem}

\subsubsection{Cele projektu}

\begin{itemize}
	\item Wdrożenie aplikacji do zarządzania dniem pracy specjalistów
	i~przypisywania im sprzętu rehabilitacyjnego na określoną porę,
	\begin{itemize}
		\item 30\% zwiększenie efektywności pracy poprzez usprawnienie systemu przypisywania specjalistom dostępu do sprzętu rehabilitacyjnego, maksymalnie pokrywając ich czas pracy zajęciami,
	\end{itemize}
	\item Wdrożenie systemu do rejestracji pacjentów,
	\begin{itemize}
		\item 50\% zmniejszenie kosztów obsługi rejestracji pacjentów, poprzez automatyzację tej procedury, co umożliwi redukcję etatów pracowników recepcji.
	\end{itemize}
\end{itemize}

\subsubsection{Użytkownicy produktu}

\begin{itemize}
	\item Klient
	\begin{itemize}
		\item Obsługa zakładu rehabilitacyjnego, administrująca czasem pracy specjalistów jak i~dostępem do sal.
	\end{itemize}
	\item Użytkownicy
	\begin{itemize}
			\item Wszyscy potencjalni klienci rejestrujący się na zajęcia rehabilitacyjne.
	\end{itemize}
\end{itemize}

\subsection{Ograniczenia projektu}

\subsubsection{Ograniczenia wynikające z natury i~okoliczności projektu}

\begin{itemize}
	\item Wszystkie dane klientów muszą podlegać ochronie zgodnie z ustawą o~ochronie danych osobowych (Dz.U. 1997 Nr 133 poz. 883),
	\item Aplikacja ma zostać wykonana w formie aplikacji webowej,
	\item Do działania aplikacji konieczne jest połączenie z internetem,
	\item Wszystkie prace projektowe powinny zostać zaplanowane tak, aby umożliwić uruchomienie i~pełne funkcjonowanie systemu do 1~grudnia~2017~roku,
Brak funduszy, ponieważ jest to projekt studencki.
	%TODO
	\item Dodałbym jeszcze ograniczenie związane z technologią czyli Yii2 jak mniemam. Zbychu bardziej sie znasz to możesz ładnie sformułować.
\end{itemize}

\subsubsection{Konwencje nazewnicze i~definicje}
%TODO !!!!! Tutaj trzeba wstawić wszystkie akronimy i skróty myślowe które popełnimy w tym dokumencie !!!
\begin{itemiz}
	\item \textit{Aplikacja webowa} - program znajdujący się na serwerze komunikujący się z użytkownikiem za pomocą przeglądarki internetowej,
	\item \textit{Przeglądarka internetowa} - program służący do wyświetlania aplikacji webowych,
	\item \textit{Logowanie} - proces autoryzacji użytkownika polegający podaniu identyfikatora i~hasła,
\end{itemize}

\subsubsection{Fakty i~założenia powiązane z~projektem}
\begin{itemize}
	\item Witryna ma być zaprojektowana w taki sposób, aby nie sprawiać trudności podczas rejestracji pacjentów.
	\item Należy założyć, że z~witryny mogą korzystać zarówno osoby starsze jak i~młodsze, interfejs powinien być intuicyjny aby obsłużyć go mogły osoby niedoświadczone jak i~aby osoby obyte z~tego typu narzędziami mogły szybko wykonać chciane operacje. %TODO paskudny język
\end{itemize}
