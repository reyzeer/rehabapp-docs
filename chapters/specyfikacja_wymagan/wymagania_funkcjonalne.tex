\subsection{Wymagania funkcjonalne}

\newcolumntype{L}[1]{>{\raggedright\bfseries}p{#1}}
\newcolumntype{R}[1]{>{\raggedright}p{#1}}

\def\funcreq#1#2#3#4#5#6#7#8#9{%
	\def\funcreqcontinued##1{%
		\vspace{0.2cm}
		\begin{tabularx}{\textwidth}{|L{4cm}|R{11.1cm}|}
			\hline
			Nr wymagania & #1 \tabularnewline \hline
			Nazwa & #2 \tabularnewline \hline
			Opis & #3 \tabularnewline \hline
			Przesłanka & #4 \tabularnewline \hline
			Ograniczenia i warunki & #5 \tabularnewline \hline
			Dane wejściowe & #6 \tabularnewline \hline
			Źródło danych & #7 \tabularnewline \hline
			Wynik & #8 \tabularnewline \hline
			Priorytet & #9 \tabularnewline \hline
			Kryterium do spełnienia & ##1 \tabularnewline \hline
		\end{tabularx}
		\vspace{0.2cm}
	}%
	\funcreqcontinued%
}

%Przyklad: \funcreq{1}{Nazwa}{Opis}{Przesłanka}{Ograniczenia i warunki}{Dane wejściowe}{Źródło danych}{Wynik}{Priorytet}{Kryterium do spełnienia}

%\funcreq{
%		1
%	}{
%		Nazwa
%	}{
%		Opis
%	}{
%		Przesłanka
%	}{
%		Ograniczenia i warunki
%	}{
%		Dane wejściowe
%	}{
%		Źródło danych
%	}{
%		Wynik
%	}{
%		Priorytet
%	}{
%		Kryterium do spełnienia
%}

\funcreq{
	1
}{
	Logowanie do panelu administracyjnego
}{
	Logowanie do systemu w celu autoryzacji uprawnienia użytkownika korzystającego z panelu administracyjnego
}{
	Prawo do korzystania z panelu administracyjnego może mieć tylko uprawniona osoba
}{
	Weryfikacja użytkownika powinna zapewnić poufność i bezpieczeństwo 
}{
	Email oraz hasło
}{
	Użytkownik
}{
	Potwierdzenie tożsamości użytkownika i przyznanie dostępu do panelu administracyjnego
}{
	Wysoki
}{
	Funkcja została zaimplementowana zgodnie z założeniami i działa poprawnie.
}