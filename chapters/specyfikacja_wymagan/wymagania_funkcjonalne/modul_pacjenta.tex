\subsubsection{Moduł pacjenta}

\funcreq{
	27
}{
	Rejestracja pacjenta
}{
	System ma posiadać funkcjonalność rejestracji dla pacjenta.\\
	Rejestracja ma zostać potwierdzona po przez otworzenie przez użytkownika linku, który zostanie wysłany na jego skrzynkę mailową po wysłaniu poprawnie wypełnionego formularza rejestracyjnego.
}{
	System ma posiadać funkcjonalność rejestracji dla pacjenta. Co umożliwi utworzenie konta do którego zostaną przypisane podstawowe informacje, a także dzięki, któremu użytkownik będzie miał dostęp do umówionych wizyt i~ich historii.
}{
	Użytkownik nie jest zalogowany.\\
	Nie może być dwóch użytkowników z~tym samym adresem email.
}{
	Imię, nazwisko, email, numer telefonu, hasło i powtórz hasło.
}{
	Użytkownik
}{
	Dodanie konta pacjenta do bazy danych.
}{
	Wysoki
}{
	Funkcja została zaimplementowana zgodnie z~założeniami i~działa poprawnie.
}

\funcreq{
	28
}{
	Logowanie pacjenta
}{
	System ma posiadać funkcjonalność logowania dla pacjenta.
}{
	Umożliwia dostęp do rejestracji i~prywatnej historii pacjentowi.
}{
	Użytkownik nie jest zalogowany do panelu administracyjnego.\\
	Weryfikacja użytkownika powinna zapewnić poufność i~bezpieczeństwo procesu.
}{
	Email oraz hasło
}{
	Użytkownik
}{
	Potwierdzenie tożsamości użytkownika i~przyznanie dostępu do opcji dostępnych po zalogowaniu w~module dla pacjenta.
}{
	Wysoki
}{
	Funkcja została zaimplementowana zgodnie z~założeniami i~działa poprawnie.
}

\funcreq{
	29
}{
	Resetowanie hasła do konta pacjenta
}{
	System umożliwia zmianę hasła do konta pacjenta. Podanie w~formularz poprawnego adresu email przypisanego do tego konta, ma skutkować wysłaniem na wskazany adres wiadomości z~linkiem. Odnośnik ten ma prowadzić na podstronę ustawienia nowego hasła.
}{
	W~razie utraty hasła przez pacjenta, ma umożliwić odzyskanie dostępu do konta.
}{
	Użytkownik nie jest zalogowany do panelu administracyjnego.\\
	Bezpieczeństwo procesu.
}{
	Email
}{
	Użytkownik
}{
	Zmiana hasła do konta pacjenta.
}{
	Wysoki
}{
	Funkcja została zaimplementowana zgodnie z~założeniami i~działa poprawnie.
}

\funcreq{
	30
}{
	Zmiana hasła
}{
	System ma umożliwić zmianę hasła do konta pacjenta.\\
	Zmiana hasła ma zostać potwierdzona linkiem wysyłanym na adres email przypisany do konta.
}{
	Pozwala odebrać dostęp osobą niepowołanym, które weszły w~posiadanie danych do konta.
}{
	Użytkownik jest zalogowany do panelu administracyjnego.\\
	Bezpieczeństwo procesu.
}{
	Aktualne hasło, nowe hasło i~potwierdzenie nowego hasła.
}{
	Użytkownik
}{
	Zmiana hasła do konta pacjenta.
}{
	Wysoki
}{
	Funkcja została zaimplementowana zgodnie z~założeniami i~działa poprawnie.
}

\funcreq{
	31
}{
	Zmiana adresu email
}{
	System ma umożliwić zmianę adresu email do konta pacjenta.
}{
	W razie utraty dostępu lub bezpieczeństwa przypisanego adresu email do konta pozwala na jego zmianę.
}{
	Użytkownik jest zalogowany do panelu administracyjnego.\\
	Bezpieczeństwo procesu.
}{
	Aktualne hasło i~email.
}{
	Użytkownik
}{
	Zmiana adresu email przypisanego do konta pacjenta.
}{
	Wysoki
}{
	Funkcja została zaimplementowana zgodnie z~założeniami i~działa poprawnie.
}

\funcreq{
	32
}{
	Wylogowanie z~modułu pacjenta
}{
	System ma umożliwić wylogowanie z~modułu pacjenta.
}{
	Ma to uniemożliwić dostęp do poufnych danych osobą trzecim, które uzyskały dostęp do komputera z~którego wykonywane są logowania.
}{
	Użytkownik jest zalogowany do modułu pacjenta.\\
	Bezpieczeństwo procesu.
}{
	-
}{
	Użytkownik
}{
	Odebranie dostęp do funkcjonalności po zalogowaniu panelu pacjenta.
}{
	Wysoki
}{
	Funkcja została zaimplementowana zgodnie z~założeniami i~działa poprawnie.
}

\funcreq{
	33
}{
	Wyszukiwanie
}{
	System ma umożliwić wyszukiwanie zabiegów i~lekarzy.\\
	Wyszukiwanie ma się odbywać przez wprowadzenie frazy pasującej do nazwy zabiegu lub specjalizacji lekarza lub imię i~nazwiska lekarza.\\
	W~wynikach ma wyświetlić listę lekarzy pasujących do wprowadzonych danych.\\
	Wybranie danego lekarza ma przenieść do jego publicznego profilu.
}{
	Ma to umożliwić znalezienie właściwego lekarza. 
}{
	-
}{
	Szukana fraza
}{
	Użytkownik
}{
	Wyświetlenie zgodnych wyników.
}{
	Wysoki
}{
	Funkcja została zaimplementowana zgodnie z~założeniami i~działa poprawnie.
}

\funcreq{
	34
}{
	Profil lekarza
}{
	Wyświetla opis lekarza na podstawie danych wprowadzonych w~panelu administracyjnym.\\
	Wyświetla kalendarz z~terminami wizyt lekarza.\\
	Przy terminach wizyt mają pojawiać się nazwy wykonywanych w~tedy zabiegów i~ich ceny.\\
	Ma pojawiać się informacja czy termin jest przeszły, dostępny lub zajęty.\\
	Dostępne terminy mają być aktywne do kliknięcia.
}{
	Ma to umożliwić sprawdzenie lekarza przez pacjenta i~przeglądnięcie jego terminarza, a także w razie chęci skorzystania z~zabiegu przejście dalej.
}{
	Użytkownik jest zalogowany do modułu pacjenta.
}{
	Lekarz
}{
	Użytkownik
}{
	Wyświetlenie opisu lekarza z~~jego kalendarzem.
}{
	Wysoki
}{
	Funkcja została zaimplementowana zgodnie z~założeniami i~działa poprawnie.
}

\funcreq{
	35
}{
	Umówienie na wizytę
}{
	Kliknięcie w~odnośnik w~kalendarzu z~wymagania 31, w przypadku niezalogowanego użytkownika przenosi do strony logowania.\\
	W~przypadku zalogowanego wyświetla pytanie z potwierdzeniem wizyty.
}{
	Ma to umożliwić rejestrację na wizytę.
}{
	Użytkownik jest zalogowany do modułu pacjenta.\\
	Dany termin musi być w~przyszłości, nie wcześniej niż za 3h.\\
	Dany termin musi być wolny.
}{
	Termin
}{
	Użytkownik
}{
	Zapisanie na termin.
}{
	Wysoki
}{
	Funkcja została zaimplementowana zgodnie z~założeniami i~działa poprawnie.
}

\funcreq{
	36
}{
	Wizyty
}{
	Użytkownik ma możliwość przeglądnięcia historii wizyt oraz wyświetlenia ich opisu.\\
	W historii mają być dostępne również przyszłe wizyty.
}{
	Ma to umożliwić wgląd w historię wizyt.
}{
	Użytkownik jest zalogowany do modułu pacjenta.
}{
	Szukana fraza
}{
	Użytkownik
}{
	Wyświetlenie historii wizyt.
}{
	Średni
}{
	Funkcja została zaimplementowana zgodnie z~założeniami i~działa poprawnie.
}

\funcreq{
	37
}{
	Odwołanie wizyty
}{
	Użytkownik ma możliwość odwołania wizyty.
}{
	Jeśli pacjent nie będzie mógł pojawić się na wizycie powinien móc z niej zrezygnować.
}{
	Użytkownik jest zalogowany do modułu pacjenta.\\
	Wizyta z przyszłości, nie wcześniej niż za 3h.
}{
	Wizyta.
}{
	Użytkownik
}{
	Odwołanie wizyty.
}{
	Średni
}{
	Funkcja została zaimplementowana zgodnie z~założeniami i~działa poprawnie.
}
