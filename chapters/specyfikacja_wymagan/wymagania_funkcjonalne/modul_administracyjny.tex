\subsubsection{Moduł administracyjny}

\funcreq{
	1
}{
	Logowanie do panelu administracyjnego
}{
	System ma posiadać funkcjonalność logowania do panelu administracyjnego, przy pomocy konta administratora.
}{
	Uniemożliwienie dostępu osobą niepowołanym.
}{
	Użytkownik nie jest zalogowany do panelu administracyjnego.\\
	Weryfikacja użytkownika powinna zapewnić poufność i~bezpieczeństwo procesu.
}{
	Email oraz hasło
}{
	Użytkownik
}{
	Potwierdzenie tożsamości użytkownika i~przyznanie dostępu do panelu administracyjnego.
}{
	Wysoki
}{
	Funkcja została zaimplementowana zgodnie z~założeniami i~działa poprawnie.
}

\funcreq{
	2
}{
	Resetowanie hasła do konta administratora
}{
	System umożliwia zmianę hasła do konta administratora. Podanie w~formularz poprawnego adresu email przypisanego do tego konta, ma skutkować wysłaniem na wskazany adres wiadomości z~linkiem. Odnośnik ten ma prowadzić na podstronę ustawienia nowego hasła.
}{
	W~razie utraty hasła przez administratora systemu, ma umożliwić odzyskanie dostępu do panelu administracyjnego.
}{
	Użytkownik nie jest zalogowany do panelu administracyjnego.\\
	Bezpieczeństwo procesu.
}{
	Email
}{
	Użytkownik
}{
	Zmiana hasła do konta administratora.
}{
	Wysoki
}{
	Funkcja została zaimplementowana zgodnie z~założeniami i~działa poprawnie.
}

\funcreq{
	3
}{
	Zmiana hasła
}{
	System ma umożliwić zmianę hasła do konta administracyjnego.\\
	Zmiana hasła ma zostać potwierdzona linkiem wysyłanym na adres email przypisany do konta.
}{
	Pozwala odebrać dostęp osobą niepowołanym, które weszły w~posiadanie danych do konta.
}{
	Użytkownik jest zalogowany do panelu administracyjnego.\\
	Bezpieczeństwo procesu.
}{
	Aktualne hasło, nowe hasło i~potwierdzenie nowego hasła.
}{
	Użytkownik
}{
	Zmiana hasła do konta administratora.
}{
	Wysoki
}{
	Funkcja została zaimplementowana zgodnie z~założeniami i~działa poprawnie.
}

\funcreq{
	4
}{
	Zmiana adresu email
}{
	System ma umożliwić zmianę adresu email do konta administracyjnego.
}{
	W razie utraty dostępu lub bezpieczeństwa przypisanego adresu email do konta pozwala na jego zmianę.
}{
	Użytkownik jest zalogowany do panelu administracyjnego.\\
	Bezpieczeństwo procesu.
}{
	Aktualne hasło i~email.
}{
	Użytkownik
}{
	Zmiana adresu email przypisanego do konta administratora.
}{
	Wysoki
}{
	Funkcja została zaimplementowana zgodnie z~założeniami i~działa poprawnie.
}

\funcreq{
	5
}{
	Wylogowanie z~panelu administracyjnego
}{
	System ma umożliwić wylogowanie z~panelu administracyjnego.
}{
	Ma to uniemożliwić dostęp do panelu osobą trzecim, które uzyskały dostęp do komputera z~którego wykonywane są logowania do niego.
}{
	Użytkownik jest zalogowany do panelu administracyjnego.\\
	Bezpieczeństwo procesu.
}{
	-
}{
	Użytkownik
}{
	Odebranie dostęp do funkcjonalności panelu administracyjnego.
}{
	Wysoki
}{
	Funkcja została zaimplementowana zgodnie z~założeniami i~działa poprawnie.
}

\funcreq{
	6
}{
	Dodanie sprzętu
}{
	System umożliwia dodanie nowego sprzętu rehabilitacyjnego do bazy danych.\\
}{
	Aby móc administrować zasobami konieczne jest ich dodanie do bazy danych w~systemie.
}{
	Użytkownik jest zalogowany do panelu administracyjnego.\\
	Nie może być dwóch sprzętów o~tej samej nazwie.\\
	Ilość musi być większa od zera.
}{
	Nazwa sprzętu i~ilość.
}{
	Użytkownik
}{
	Dodanie nowego sprzętu do bazy danych.
}{
	Wysoki
}{
	Funkcja została zaimplementowana zgodnie z~założeniami i~działa poprawnie.
}

\funcreq{
	7
}{
	Lista sprzętu
}{
	System umożliwia przeglądanie oraz filtorwanie sprzętu po nazwie, ilości, nazwie zabiegu, imieniu~i nazwisku lekarza.
}{
	System musi umożliwiać przeglądanie dostępnego w~bazie danych sprzętu.
}{
	Użytkownik jest zalogowany do panelu administracyjnego.
}{
	Nazwa sprzętu (opcjonalne), ilość(opcjonlane), nazwa zabiegu (opcjonalne), imie i~naziwsko lekarza.
}{
	Użytkownik
}{
	Wyświetlenie listy sprzętów, które spełnia kryteria z~filtrów.
}{
	Wysoki
}{
	Funkcja została zaimplementowana zgodnie z~założeniami i~działa poprawnie.
}

\funcreq{
	8
}{
	Podgląd sprzętu
}{
	System umożliwia podgląd sprzętu.
}{
	System musi umożliwiać podgląd nazwy, ilości sprzętu.\\
	System ma także wyświetlić listę zabiegów wykorzystujących ten sprzęt.\\
	System ma także wyświetlić listę lekarzy wykorzystujących ten sprzęt.
}{
	Użytkownik jest zalogowany do panelu administracyjnego.
}{
	-
}{
	Użytkownik
}{
	Wyświetlenie opis sprzętów oraz listę powiązanych z~nim zabiegów i~lekarzy.
}{
	Wysoki
}{
	Funkcja została zaimplementowana zgodnie z~założeniami i~działa poprawnie.
}

\funcreq{
	9
}{
	Edycja sprzętu
}{
	System umożliwia edycję informacji o~sprzęcie.
}{
	System ma umożliwić zmianę błędnej nazwy lub ilości sprzętu.
}{
	Użytkownik jest zalogowany do panelu administracyjnego.\\
	Nie może być dwóch sprzętów o~tej samej nazwie.\\
	Ilość musi być większa od zera.
}{
	Nazwa sprzętu i~ilość.
}{
	Użytkownik
}{
	Zmiana nazwy i~ilości sprzętu.
}{
	Wysoki
}{
	Funkcja została zaimplementowana zgodnie z założeniami i działa poprawnie.
}

\funcreq{
	10
}{
	Usuwanie sprzętu
}{
	System ma umożliwić usuwanie sprzętu z~bazy danych.\\
	Usuwanie ma się odbywać po przez nadanie rekordowi w~bazie danych statusu usuniętego, ukrywając jego dostępność w~całym systemie, ale nie usuwając całkowicie informacji o~nim z~bazy danych.\\
	Podczas próby usunięcia sprzętu do którego jest przypisany co najmniej jeden zabieg musi zostać wyświetlony stosowny komunikat.
}{
	W~razie zniszczenia, bądź rezygnacji z~danego sprzętu w~zakładzie musi istnieć możliwość usunięcia.
}{
	Użytkownik jest zalogowany do panelu administracyjnego.\\
	Do sprzętu nie może być przypisany żaden zabieg.
}{
	Sprzęt
}{
	Użytkownik
}{
	Ukrycie dostępności sprzętu w~całym systemie na zawsze.
}{
	Wysoki
}{
	Funkcja została zaimplementowana zgodnie z założeniami i działa poprawnie.
}

\funcreq{
	11
}{
	Dodanie zabiegu
}{
	System umożliwia dodanie nowego zabiegu rehabilitacyjnego do bazy danych.\\
}{
	Aby móc administrować zasobami konieczne jest ich dodanie do bazy danych w~systemie.
}{
	Użytkownik jest zalogowany do panelu administracyjnego.\\
	Nie może być dwóch zabiegów o~tej samej nazwie.
}{
	Nazwa zabiegu, czas trwania (wybór opcji: 30m, 1h, 1.5h, 2h), cena i~wymagane sprzęty (opcjonalne).
}{
	Użytkownik
}{
	Dodanie nowego zabiegu do bazy danych.
}{
	Wysoki
}{
	Funkcja została zaimplementowana zgodnie z~założeniami i~działa poprawnie.
}

\funcreq{
	12
}{
	Lista zabiegów
}{
	System umożliwia przeglądanie oraz filtorwanie zabiegów po nazwie, czasie trwania, cenie, nazwie sprzętu, imieniu i~nazwisku lekarza.
}{
	System musi umożliwiać przeglądanie dostępnych w~bazie danych zabiegów.
}{
	Użytkownik jest zalogowany do panelu administracyjnego.
}{
	Nazwa zabiegu (opcjonalne), czas trwania (wybór opcji: 30m, 1h, 1.5h, 2h), cena (opcjonalne), nazwa sprzętu (opcjonlane), imie i~naziwsko lekarza.
}{
	Użytkownik
}{
	Wyświetlenie listy zabiegów, które spełnia kryteria z~filtrów.
}{
	Wysoki
}{
	Funkcja została zaimplementowana zgodnie z~założeniami i~działa poprawnie.
}

\funcreq{
	13
}{
	Podgląd zabiegu
}{
	System umożliwia podgląd zabiegu.
}{
	System musi umożliwiać podgląd nazwy i ceny.\\
	System ma także wyświetlić listę sprzętów wykorzystywanych do tego zabiegu.\\
	System ma także wyświetlić listę lekarzy wykonujących ten zabieg.
}{
	Użytkownik jest zalogowany do panelu administracyjnego.
}{
	-
}{
	Użytkownik
}{
	Wyświetlenie opis zabiegu oraz listę powiązanych z~nim sprzętów i~lekarzy.
}{
	Wysoki
}{
	Funkcja została zaimplementowana zgodnie z~założeniami i~działa poprawnie.
}

\funcreq{
	14
}{
	Edycja zabiegu
}{
	System umożliwia edycję informacji o~zabiegu.
}{
	System ma umożliwić zmianę błędnej nazwy, czasu trwania, ceny lub powiązanego sprzętu.
}{
	Użytkownik jest zalogowany do panelu administracyjnego.\\
	Nie może być dwóch zabiegów o~tej samej nazwie.
}{
	Nazwa zabiegu, czas trwania (wybór opcji: 30m, 1h, 1.5h, 2h), cena i~wymagane sprzęty (opcjonalne).
}{
	Użytkownik
}{
	Zmiana nazwy lub przypisanego sprzętu do zabiegu w~bazy danych.
}{
	Wysoki
}{
	Funkcja została zaimplementowana zgodnie z~założeniami i~działa poprawnie.
}

\funcreq{
	15
}{
	Usuwanie zabiegu
}{
	System ma umożliwić usuwanie zabiegu z~bazy danych.\\
	Usuwanie ma się odbywać po przez nadanie rekordowi w~bazie danych statusu usuniętego, ukrywając jego dostępność w~całym systemie, ale nie usuwając całkowicie informacji o~nim z~bazy danych.\\
	Podczas próby usunięcia zabiegu do którego jest przypisany co najmniej jeden lekarz musi zostać wyświetlony stosowny komunikat.
}{
	W~razie rezygnacji z~wykonywanie danego zabiegu w~zakładzie musi istnieć możliwość usunięcia go z~systemu.
}{
	Użytkownik jest zalogowany do panelu administracyjnego.\\
	Do zabiegu nie może być przypisany żaden lekarz.
}{
	Zabieg
}{
	Użytkownik
}{
	Ukrycie dostępności zabiegu w~całym systemie na zawsze.
}{
	Wysoki
}{
	Funkcja została zaimplementowana zgodnie z założeniami i działa poprawnie.
}

\funcreq{
	16
}{
	Dodanie lekarza
}{
	System umożliwia dodanie nowego lekarza do bazy danych.\\
	Po dodaniu lekarza do bazy danych, na wpisany w~formularzu adres email wysyłany jest link, który pozwoli na ustawinie hasła do konta.
}{
	Aby móc administrować zasobami konieczne jest ich dodanie do bazy danych w~systemie.
}{
	Użytkownik jest zalogowany do panelu administracyjnego.\\
	Nie może być dwóch lekarzy z~tym samym adresem email.
}{
	Imię lekarza, nazwisko, email, specjalizacja lekarza, opis, zdjęcie (opcjonlane), lista wykonywanych zabiegów (conajmniej jeden).
}{
	Użytkownik
}{
	Dodanie nowego zabiegu do bazy danych.
}{
	Wysoki
}{
	Funkcja została zaimplementowana zgodnie z~założeniami i~działa poprawnie.
}

\funcreq{
	17
}{
	Lista lekarzy
}{
	System umożliwia przeglądanie oraz filtorwanie lekarzy po imieniu, nazwisku i wykonywanych zabiegach.
}{
	System musi umożliwiać przeglądanie dostępnych w~bazie danych lekarzy.
}{
	Użytkownik jest zalogowany do panelu administracyjnego.
}{
	Imię (opcjonalne), nazwisko (opcjonalne), email (opcjonalne), nazwa zabiegu(opcjonalne) i sprzęt(opcjonalne).
}{
	Użytkownik
}{
	Wyświetlenie listy lekarzy, którzy spełniają kryteria z~filtrów.
}{
	Wysoki
}{
	Funkcja została zaimplementowana zgodnie z~założeniami i~działa poprawnie.
}

\funcreq{
	18
}{
	Podgląd lekarza
}{
	System umożliwia podgląd lekarza.
}{
	System musi umożliwiać podgląd lekarza.\\
	System ma także wyświetlić listę sprzętów wykorzystywanych przez lekarza.\\
	System ma także wyświetlić listę zabiegów wykonywanych przez lekarza.
}{
	Użytkownik jest zalogowany do panelu administracyjnego.
}{
	-
}{
	Użytkownik
}{
	Wyświetlenie opis lekarza oraz listy powiązanych z~nim sprzętów i~zabiegów.
}{
	Wysoki
}{
	Funkcja została zaimplementowana zgodnie z~założeniami i~działa poprawnie.
}

\funcreq{
	19
}{
	Edycja lekarza
}{
	System umożliwia edycję informacji o~lekarzu.
}{
	System ma umożliwić zmianę błędnych danych lub powiązanych zabiegów.
}{
	Użytkownik jest zalogowany do panelu administracyjnego.\\
	Nie może być dwóch lekarzy z~tym samym adresem email.
}{
	Imię lekarza, nazwisko, email, specjalizacja lekarza, opis, zdjęcie (opcjonlane), lista wykonywanych zabiegów (conajmniej jeden).
}{
	Użytkownik
}{
	Zmiana danych lekarza w~bazy danych.
}{
	Wysoki
}{
	Funkcja została zaimplementowana zgodnie z~założeniami i~działa poprawnie.
}

\funcreq{
	20
}{
	Zmiana hasła lekarza
}{
	System umożliwia wysłanie odnośnika na adres email przypisany do konta lekarza umożliwiając zmianę hasła.\\
	Po otwarciu linku pojawia się formularz do wprowadzenia nowego hasła.
}{
	Pozwala odebrać dostęp osobą niepowołanym, które weszły w~posiadanie danych do konta.
}{
	Dla panelu administracyjnego: Użytkownik jest zalogowany do panelu administracyjnego.
	Dla linku zmiany hasła: Użytkownik jest nie zalogowany do panelu administracyjnego.
}{
	Lekarz | Nowe hasło i potwierdzenie nowego hasła.
}{
	Użytkownik
}{
	Zmiana hasła do konta lekarza w~bazy danych.
}{
	Wysoki
}{
	Funkcja została zaimplementowana zgodnie z~założeniami i~działa poprawnie.
}

\funcreq{
	21
}{
	Usuwanie lekarza
}{
	System ma umożliwić usuwanie lekarza z~bazy danych.\\
	Usuwanie ma się odbywać po przez nadanie rekordowi w~bazie danych statusu usuniętego, ukrywając jego dostępność w~całym systemie, ale nie usuwając całkowicie informacji o~nim z~bazy danych.\\
	Podczas próby usunięcia lekarza do którego jest umówiona co najmniej jedna wizyta musi zostać wyświetlony stosowny komunikat.
}{
	W~razie zakończenia pracy lekarza w~zakładzie musi istnieć możliwość usunięcia go z~listy dostępnych lekarzy.
}{
	Użytkownik jest zalogowany do panelu administracyjnego.\\
	Lekarz nie ma umówionych wizyt.
}{
	Lekarz
}{
	Użytkownik
}{
	Ukrycie dostępności lekarza w~całym systemie na zawsze.
}{
	Wysoki
}{
	Funkcja została zaimplementowana zgodnie z założeniami i działa poprawnie.
}

\funcreq{
	22
}{
	Dodanie terminów wizyt lekarzowi
}{
	System ma umożliwić dodanie terminów wizyt lekarzowi.\\
}{
	Dodanie terminów przyjęć na zabiegi będzie źródłem danych o~godzinach zabiegów dla pacjentów oraz czasie zajętości sprzętu.
}{
	Użytkownik jest zalogowany do panelu administracyjnego.\\
	Lekarz nie ma w tym czasie innych terminów zabiegów.\\
	Sprzęt potrzebny do wykonania zabiegów jest dostępny.
}{
	Lekarz, zabieg, dni tygodnia i przedział godzin w~ciągu dnia w~których odbywa się zabieg u tego lekarza i przedział dat dla których mają zostać dodane terminy zabiegów.
}{
	Użytkownik
}{
	Dodanie terminów zabiegów do bazy danych.
}{
	Wysoki
}{
	Funkcja została zaimplementowana zgodnie z założeniami i działa poprawnie.
}

\funcreq{
	23
}{
	Wyświetlanie kalendarza
}{
	System ma umożliwić wyświetlenie kalendarza wizyt dla danego lekarza.
}{
	Musi istnieć podgląd kalendarza wizyt, aby mieć wgląd w grafik pracy lekarza.
}{
	Użytkownik jest zalogowany do panelu administracyjnego.
}{
	Lekarz
}{
	Użytkownik
}{
	Wyświetlenie kalendarza lekarza.
}{
	Wysoki
}{
	Funkcja została zaimplementowana zgodnie z założeniami i działa poprawnie.
}

\funcreq{
	24
}{
	Usunięcie terminu
}{
	System ma umożliwić usunięcie dostępności terminu.\\
	W razie gdy na dany termin jest umówiony pacjent ma pojawić się stosowny komunikat.
}{
	Potrzebne jest narzędzie, które pozwoli odowałać terminy zabiegów.
}{
	Użytkownik jest zalogowany do panelu administracyjnego.\\
	Na dany termin nie może być umówiony pacjent.\\
	Termin jeszcze nie upłynął.
}{
	Termin
}{
	Użytkownik
}{
	Usunięcie terminu z kalendarza.
}{
	Wysoki
}{
	Funkcja została zaimplementowana zgodnie z założeniami i działa poprawnie.
}

\funcreq{
	25
}{
	Odwołanie terminu
}{
	System ma umożliwić odwołanie wizyty.\\
	Pacjent ma zostać powiadomiony o~odwołaniu jego wizyty mailem.
}{
	Musi istnieć narzędzie pozwalające na odwołanie wizyt w~razie braku możliwości ich realizacji.
}{
	Użytkownik jest zalogowany do panelu administracyjnego.\\
	Na dany termin jest zapisany pacjent.\\
	Termin jeszcze nie upłynął.
}{
	Termin
}{
	Użytkownik
}{
	Odwołanie umówionego zabiegu.
}{
	Wysoki
}{
	Funkcja została zaimplementowana zgodnie z założeniami i działa poprawnie.
}

\funcreq{
	26
}{
	Podsumowanie terminu
}{
	Po odbyciu się wizyty na umówiony termin, ma istnieć możliwość jej potwierdzenia, że się odbyła oraz dodania krótkiej opcjonalnej informacji o jej przebiegu.
}{
	Musi istnieć miejsce do gromadzenia historii przebiegu zabiegów.
}{
	Użytkownik jest zalogowany do panelu administracyjnego.\\
	Na dany termin jest zapisany pacjent.\\
	Termin upłynął.
}{
	Termin, potwierdzenie/zaprzeczanie odbycia zabiegu i opis (opcjonalne).
}{
	Użytkownik
}{
	Dodanie informacji podsumowującej miniony umówiony zabieg.
}{
	Wysoki
}{
	Funkcja została zaimplementowana zgodnie z założeniami i działa poprawnie.
}
