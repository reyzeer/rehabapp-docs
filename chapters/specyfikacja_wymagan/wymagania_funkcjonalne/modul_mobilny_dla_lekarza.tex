\subsubsection{Moduł mobilny dla lekarza}

\funcreq{
	38
}{
	Logowanie lekarza
}{
	System ma posiadać funkcjonalność logowania dla lekarza.
}{
	Umożliwia dostęp do kalendarza lekarza i~wprowadzania w~nim zmian.
}{
	Użytkownik nie jest zalogowany do aplikacji lekarza.\\
	Weryfikacja użytkownika powinna zapewnić poufność i~bezpieczeństwo procesu.
}{
	Email oraz hasło
}{
	Użytkownik
}{
	Potwierdzenie tożsamości użytkownika i~przyznanie dostępu do opcji dostępnych po zalogowaniu w~aplikacji dla lekarza.
}{
	Wysoki
}{
	Funkcja została zaimplementowana zgodnie z~założeniami i~działa poprawnie.
}

\funcreq{
	39
}{
	Wylogowanie z~aplikacji lekarza
}{
	System ma umożliwić wylogowanie z~aplikacji lekarza.
}{
	Ma to uniemożliwić dostęp do poufnych danych osobą trzecim, które uzyskały dostęp do komputera z~którego wykonywane są logowania.
}{
	Użytkownik jest zalogowany do aplikacji lekarza.\\
	Bezpieczeństwo procesu.
}{
	-
}{
	Użytkownik
}{
	Odebranie dostęp do funkcjonalności po zalogowaniu do aplikacji lekarza.
}{
	Wysoki
}{
	Funkcja została zaimplementowana zgodnie z~założeniami i~działa poprawnie.
}

\funcreq{
	40
}{
	Wyświetlenie kalendarza
}{

	Wyświetla kalendarz z~terminami wizyt lekarza.\\
	Przy terminach wizyt mają pojawiać się nazwy wykonywanych w~tedy zabiegów.\\
	Ma pojawiać się informacja czy termin jest przeszły, dostępny lub zajęty.\\
	Dostępne i aktywne terminy mają być aktywne do kliknięcia.
}{
	Ma to umożliwić sprawdzenie lekarzowi swojego planu zajęć.
}{
	Użytkownik jest zalogowany do aplikacji lekarza.
}{
	-
}{
	Użytkownik
}{
	Wyświetlenie kalendarza.
}{
	Wysoki
}{
	Funkcja została zaimplementowana zgodnie z~założeniami i~działa poprawnie.
}

\funcreq{
	41
}{
	Dodanie terminów wizyt przez lekarza
}{
	System ma umożliwić dodanie terminów wizyt przez apliakcję lekarzowi.\\
}{
	Dodanie terminów przyjęć na zabiegi będzie źródłem danych o~godzinach zabiegów dla pacjentów oraz czasie zajętości sprzętu.
}{
	Użytkownik jest zalogowany do aplikacji lekarza.\\
	Lekarz nie ma w tym czasie innych terminów zabiegów.\\
	Sprzęt potrzebny do wykonania zabiegów jest dostępny.
}{
	Zabieg, dni tygodnia i przedział godzin w~ciągu dnia w~których odbywa się zabieg u tego lekarza i przedział dat dla których mają zostać dodane terminy zabiegów.
}{
	Użytkownik
}{
	Dodanie terminów zabiegów do bazy danych.
}{
	Wysoki
}{
	Funkcja została zaimplementowana zgodnie z założeniami i działa poprawnie.
}

\funcreq{
	42
}{
	Odwołanie terminu
}{
	Lekarz ma możliwość odwołania wizyty.\\
	W momencie odwołania wizyty pacjent dostaje wiadomość email z adekwatną informacją.
}{
	Jeśli lekarz nie będzie mógł zrealizować wizyty powinien móc ją odwołać.
}{
	Użytkownik jest zalogowany do aplikacji lekarza.\\
	Wizyta z przyszłości, nie wcześniej niż za 3h.
}{
	Termin
}{
	Użytkownik
}{
	Odwołanie wizyty.
}{
	Wysoki
}{
	Funkcja została zaimplementowana zgodnie z~założeniami i~działa poprawnie.
}

\funcreq{
	43
}{
	Podsumowanie terminu
}{
	Po odbyciu się wizyty na umówiony termin, ma istnieć możliwość jej potwierdzenia, że się odbyła oraz dodania krótkiej opcjonalnej informacji o jej przebiegu.
}{
	Musi istnieć miejsce do gromadzenia historii przebiegu zabiegów.
}{
	Użytkownik jest zalogowany do aplikacji lekarza.\\
	Na dany termin jest zapisany pacjent.\\
	Termin upłynął.
}{
	Termin, potwierdzenie/zaprzeczanie odbycia zabiegu i opis (opcjonalne).
}{
	Użytkownik
}{
	Dodanie informacji podsumowującej miniony umówiony zabieg.
}{
	Wysoki
}{
	Funkcja została zaimplementowana zgodnie z założeniami i działa poprawnie.
}

\funcreq{
	44
}{
	Lista pacjentów
}{
	Lekarz powinien móc przeglądać listę swoich pacjentów.\\
	W ramach listy mają funkcjonować filtry wyszukiwania pacjentów po: imieniu, nazwisku lub adresie email.
}{
	Musi istnieć miejsce w~którym lekarz będzie mógł przeszukać listę pacjentów.
}{
	Użytkownik jest zalogowany do aplikacji lekarza.
}{
	Imię pacjenta (opcjonalne), nazwisko (opcjonalne) i email (opcjonalne).
}{
	Użytkownik
}{
	Wyświetlenie listy pacjentów zgodnej z~ustawieniem filtrów.
}{
	Średni
}{
	Funkcja została zaimplementowana zgodnie z założeniami i działa poprawnie.
}

\funcreq{
	45
}{
	Historia pacjenta
}{
	Lekarz ma móc przeglądać historię pacjenta.
}{
	Musi istnieć miejsce w~którym lekarz będzie mógł podejrzeć historię przyjętych pacjentów.
}{
	Użytkownik jest zalogowany do aplikacji lekarza.
}{
	Pacjent
}{
	Użytkownik
}{
	Wyświetlenie listy odbytych terminów pacjenta wraz z~podsumowaniami.
}{
	Średni
}{
	Funkcja została zaimplementowana zgodnie z założeniami i działa poprawnie.
}
