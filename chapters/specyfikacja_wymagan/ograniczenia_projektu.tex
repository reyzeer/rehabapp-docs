\subsection{Ograniczenia projektu}

\subsubsection{Ograniczenia wynikające z natury i~okoliczności projektu}

\begin{itemize}
	\item Wszystkie dane klientów muszą podlegać ochronie zgodnie z ustawą o~ochronie danych osobowych (Dz.U. 1997 Nr 133 poz. 883),
	\item Moduły aplikacji zajmujące się administracją oraz rejestracją pacjentów mają zostać wykonane w~formie aplikacji webowej,
	\item Moduł obsługiwany przez lekarzy, ma być zaimplementowany w~formie aplikacji mobilnej (ograniczenie kosztów wdrożenia sprzętu po przez wykorzystanie inteligentnych urządzeń mobilnych, które posiadają lekarze),
	\item Do działania aplikacji konieczne jest połączenie z internetem,
	\item Wszystkie prace projektowe powinny zostać zaplanowane tak, aby umożliwić uruchomienie i~pełne funkcjonowanie systemu do 1~grudnia~2017~roku,
	\item Brak funduszy, ponieważ jest to projekt studencki.
\end{itemize}

\subsubsection{Konwencje nazewnicze i~definicje}
\begin{itemize}
	%TODO
	\item \textit{Aplikacja webowa} - program obsługiwany za pomocą przeglądarki internetowej,
	\item \textit{Aplikacja mobilna} - program uruchamiany na urządzeniu mobilnym typu smartphone lub tablet,
%	\item \textit{Logowanie} - proces autoryzacji użytkownika polegający podaniu identyfikatora i~hasła,
\end{itemize}

\subsubsection{Fakty i~założenia powiązane z~projektem}
\begin{itemize}
	\item Witryna ma być zaprojektowana w taki sposób, aby nie sprawiać trudności podczas rejestracji pacjentów,
\end{itemize}
