\subsection{Wymagania pozafunkcjonalne}

\subsubsection{Wymagania estetyczne}

\nonefuncreq{
	46
}{
	Estetyczny wygląd
}{
	Interfejs strony powinien być estetyczny i intuicyjny.
}{
	Spełnienie wymagań klienta dotyczących wyglądu serwisu internetowego.
}{
	Reprezentatywna próbka osób po pięciu minutach od pierwszego kontaktu ze stroną wypełnia ankietę dotyczącą estetyki i intuicyjności serwisu. Jeśli średnia ocen nie będzie niższa niż 3.5 na 5 punktów należy przyjąć, że aplikacja spełnia wymagania.
}

\nonefuncreq{
	47
}{
	Funkcjonalność aplikacji powinna być zachowana dla różnych platform
}{
	Interfejs aplikacji powinien działać/wyglądać identycznie/estetycznie na różnych przeglądarkach oraz systemach operacyjnych.
}{
	Zwiększenie komfortu użytkowania i potencjalnej liczby użytkowników.
}{
	Każdy członek grupy testowej porusza się po serwisie przy pomocy trzech różnych przeglądarek internetowych charakterystycznych dla danego systemu operacyjnego (Safari, Google Chrome, Chromium), po czym wypełnia ankietę. Jeśli co najmniej 90\% ankiet wskaże, że aplikacje wyglądają identycznie wymaganie zostaje uznane za spełnione. 
}

\subsubsection{Wymagania wydajnościowe}

\nonefuncreq{
	48
}{
	Efektywne działanie systemu
}{
	Zwiększenie komfortu użytkowania.
}{
	Zwiększenie komfortu użytkowania i potencjalnej liczby użytkowników.
}{
	Funkcja testowa mierzy czas pierwszych 100 logowań, jeśli średni czas nie przekracza 10 sekund wymaganie zostaje uznane za spełnione. 
}

\nonefuncreq{
	49
}{
	Intuicyjność systemu
}{
	Zwiększenie komfortu użytkowania.
}{
	Zwiększenie komfortu użytkowania i potencjalnej liczby użytkowników.
}{
	Reprezentatywna próbka osób wyszukuje dwa wybrane przez siebie zabiegi oraz dokonuje na nie zapisu odmierzając czas. Jeśli średni czas nie przekracza 4 minut wymaganie zostaje uznane za spełnione.
}

\subsubsection{Wymagania prawne}

\nonefuncreq{
	50
}{
	Prawa własności klienta do projektu aplikacji
}{
	Produkt nie może być wykorzystywany ani rozpowszechniany bez wiedzy klienta.
}{
	Ochrona interesu klienta.
}{
	Każda osoba używająca aplikacji bez wiedzy klienta jest zobowiązana wypłacić odszkodowanie.
}

\subsubsection{Wymagania dotyczące nauki produktu}

\nonefuncreq{
	51
}{
	Intuicyjność obsługi systemu
}{
	Czas potrzebny do nauki sprawnego poruszania się w~serwisie nie powinien przekroczyć 15 minut.
}{
	Chęć określenia ilości czasu potrzebnej do zapoznania się z~funkcjonalnościami serwisu.
}{
	90\% testerów ma swobodnie poruszać się po aplikacji internetowej wraz ze znajomością 80\% jej funkcjonalności po czasie krótszym niż 15 minut.
}

\subsubsection{Pozostałe wymagania}

\nonefuncreq{
	52
}{
	Warunki akceptacji realizacji projektu przez klienta
}{
	W wersji finalnej produkt powinien spełniać 90\% wymagań funkcjonalnych.
}{
	Spełnienie wymagań klienta względem funkcjonalności serwisu internetowego.
}{
	Jeśli wersja finalna spełnia 90\% przyjętych wymagań funkcjonalnych można uznać wymaganie za spełnione.
}
