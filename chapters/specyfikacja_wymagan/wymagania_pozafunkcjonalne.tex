\subsection{Wymagania pozafunkcjonalne}

\subsubsection{Wymagania estetyczne}

\nonefuncreq{
	1
}{
	Estetyczny wygląd
}{
	Interfejs strony powinien być estetyczny i intuicyjny.
}{
	Spełnienie wymagań klienta dotyczących wyglądu serwisu internetowego.
}{
	Reprezentatywna próbka osób po pięciu minutach od pierwszego kontaktu ze stroną wypełnia ankietę dotyczącą estetyki i intuicyjności serwisu. Jeśli średnia ocen nie będzie niższa niż 3.5 na 5 punktów należy przyjąć, że aplikacja spełnia wymagania.
}

%TODO czy to aby na pewno poza funkcjonlane
\nonefuncreq{
	2
}{
	Funkcjonalność aplikacji powinna być zachowana dla różnych platform
}{
	Interfejs aplikacji powinien działać/wyglądać identycznie/estetycznie na różnych przeglądarkach.
}{
	Zwiększenie komfortu użytkowania i potencjalnej liczby użytkowników.
}{
	Każdy członek grupy testowej porusza się po serwisie przy pomocy trzech różnych przeglądarek internetowych charakterystycznych dla danego systemu operacyjnego (Safari, Google Chrome, Chromium), po czym wypełnia ankietę. Jeśli co najmniej 90\% ankiet wskaże, że aplikacje wyglądają identycznie wymaganie zostaje uznane za spełnione. 
}

\subsubsection{Wymagania wydajnościowe}

\nonefuncreq{
	1
}{
	-
}{
	Zwiększenie komfortu użytkowania.
}{
	Zwiększenie komfortu użytkowania i potencjalnej liczby użytkowników.
}{
	Funkcja testowa mierzy czas pierwszych 100 logowań, jeśli średni czas nie przekracza 10 sekund wymaganie zostaje uznane za spełnione. 
}

\nonefuncreq{
	2
}{
	-
}{
	Zwiększenie komfortu użytkowania.
}{
	Zwiększenie komfortu użytkowania i potencjalnej liczby użytkowników.
}{
	Reprezentatywna próbka osób wyszukuje dwa wybrane przez siebie zabiegi oraz dokonuje na nie zapisu odmierzając czas. Jeśli średni czas nie przekracza 4 minut wymaganie zostaje uznane za spełnione.
}
